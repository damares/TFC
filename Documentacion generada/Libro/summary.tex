\begin{otherlanguage}{english}

\chapter*{Summary}
\addcontentsline{toc}{chapter}{Summary}

The project of building Europe begins in 1951 with the foundation of The European Coal and Steel Community. After two devastating world wars, two new great powers raised, United States and the Soviet Union, thus pushing the European countries into the background. Bigger steps to a European union have been taken since then, e.g. the creation of the European Free Trade Association in 1960, the signing of the Merger Treaty in 1967, the creation of the European Monetary System in 1979 and the Single European Act in 1986. In the beginning, this was thought to be a method to avoid confrontation between European countries, but it ended up being a political project brought forward in 1989 with the fall of Berlin wall. It was culminated with the sign of Lisbon Treaty in 2007 and the foundation of the European Union. 

As a consequence, European universities have given shape to a new high education system valid for the whole continent. This educational frame seeks to increase compatibility and comparability of educational systems respecting their diversity. Since the signing of the Charta Magna Universitatum in 1986 to the Bologna Process in 1999, the European project becomes established by taking the European Humanism as a base of the European reform. This focus on the European Humanism triggers off a change in the educational system. What it used to be a teacher-focused curriculum becomes a student-centered one. As a consequence of this paradigm change, Spanish universities must face operational changes and a redefinition of the educative offer.

Changes as the creation of a new competence-evaluation system, the European Credit Transfer System, and the definition of a training plan, bring about the introduction of many new concepts. These concepts are not always known to the staff that has to develop these curricula. Even if the staff really knows the meaning of these terms, there can be misleading nuances that can lead to misunderstandings. 

Inappropriate, not enlightening tools are therefore used to redesign curricula. This leads to increase the difficulties of students? training plan design. These tools are not useful, reusable and not much automatic, which triggers that a small modification in curricula or the simple comprehension of why a subject is included in them, becomes a very difficult task. 

Therefore, the use of semantic tools as ontologies and automation of creation and maintenance tasks will be outlined in this paper. Ontologies allow a more precise handling of the knowledge frame by unifying it and taking ambiguities and incoherence off. 

Thus, elaboration of a formal, unique frame will be the main aim of this paper. This frame?s goal is to unify knowledge so as to all the concepts are unambiguous and every person can consult, add or modify information, even if they are not familiar to curricular structure. This unique frame guarantees that all the information included in the system is coherent with modeling knowledge. In addition to this, it will allow to improve the presentation of information, so relations between subjects, topics and curricula will be clearer. Users will be able to locate the place they have in the modeling universe, have a wider understanding of how their work and the information that is handled in the curriculum which is being modeled, match. We will therefore create a more flexible system that allows an easier adaptation of educational systems to new scenarios when normative or functional changes may occur. 

\end{otherlanguage}

%%% Local Variables: 
%%% mode: latex
%%% TeX-master: "tfc-ontologia-grado"
%%% TeX-PDF-mode: t
%%% ispell-local-dictionary: "castellano"
%%% End: 

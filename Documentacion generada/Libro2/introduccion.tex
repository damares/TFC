\chapter{Introducción}

Durante la última década del siglo XX el movimiento europeísta fue tomando consistencia a lo largo de los diferentes países, instituciones y ciudadanos europeos. Las instituciones universitarias europeas no han sido ajenas a este movimiento, y a lo largo de las dos últimas décadas han ido dando forma, mediante la firma de diferentes acuerdos y tratados, a una nueva educación superior, de carácter paneuropeo. Esta transición nos ha llevado desde un estado fragmentado y sin cohesión ninguna, hacia el marco existente en la actualidad - el Espacio Europeo de Educación Superior - que aumenta la compatibilidad y la comparabilidad de los distintos sistemas de educación, respetando siempre su diversidad.
	
%En este capítulo haremos un breve repaso histórico a través de los diferentes acuerdos y tratados firmados, que han llevado a la educación superior en europa desde un estado fragmentado y sin cohesión ninguna, hacia el marco existente en la actualidad - el Espacio Europeo de Educación Superior - que aumenta la compatibilidad y la comparabilidad de los distintos sistemas de educación, respetando siempre su diversidad. 

\section{Espacio europeo}
El Espacio Europeo de Educación Superior (EEES en adelante) es un ambicioso proyecto puesto en marcha a nivel europeo para armonizar los diferentes sistemas universitarios europeos, y dotar de una mayor agilidad a unviersidades y alumnos mediante el intercambio de ideas y personas. 

No tiene como objetivo estandarizar los diversos sistemas de educación superior sino aumentar su compatibilidad y comparabilidad. 

	\subsection{Historia}

		\subsubsection{\bfseries \itshape Charta Magna Universitatum}
		En el año 1986, la Universidad de Bolonia realiza una propuesta a las más antiguas universidades europeas de crear una carta magna europea que recoja los valores tradicionales de la universidad y que aboge por la difusión de sus bondades. Esta idea tuvo una gran acogida por las universidades, que durante una	reunión delebrada en Junio de 1987 en la propia Universidad de Bolonia a la que	asistieron más de 80 delegados de diferentes universidades europeas, eligieron una comisión de ocho miembros encargados de confeccionar la Carta Magna. Esta comisíón estaba compuesta por el Presidente de la Conferencia Europea de Rectores, los rectores de las universidades de Bolonia, Paris I, Lauven, Barcelona, el profesor D.Guiseppe Caputo (Universidad de Bolonia) y el profesor D.Manuel Nuñez Encabo (Presidente de la sub-comisión universitaria de la	Asamblea Parlamentaria del Consejo Europeo. \cite{MCO:WEB}. El documento estaba concluido en 1988, y fue ratificado por todos los rectores asistentes a la celebración del nonacentésimo aniversario de la fundación de la universidad de bolonia.

		En esa Carta Magna \cite{UNIBOL:MCU-98} se considera que:
		\begin{itemize}
			\item El porvenir de la humanidad depende del desarrollo cultural, ciéntífico y técnico y que el epicentro de este desarrollo son las universidades.
			\item La obligación que la universidad contrae con la humanidad de difundir ese conocimiento para las nuevas generaciones, exige de la sociedad un esfuerzo adicional en la formación de sus ciudadanos.
			\item La universidad debe ser garante de la educación y formación de las generaciones venideras de modo que éstas contribuyan al equilibrio del entorno natural y de la vida.
		\end{itemize}
	
		Con estos hechos y objetivos, la Carta Magna proclama los cuatro principios fundamentales que sustentan la vocación de la universidad. Estos principios son:
		\begin{itemize}
			\item Independencia moral y científica frente al poder político, económico e ideológico.
			\item Indivisibilidad entre actividad docente y actividad investigadora.
			\item Libertad de investigación, enseñanza y formación. La universidad es un lugar de encuentro entre personas con la capacidad de transmitir el saber y ampliarlo con los medios puestos a su alcance para la investigación y el desarrollo (profesores) y personas que tienen el derecho, la capacidad y la voluntad de enriquecerse con ello.
			\item Eliminación de cualquier frontera geográfica o política y fomento del conocimiento intercultural.
		\end{itemize}
	
		Buscando cumplir esos muy ilustres objetivos, finaliza la carta magna manifestando la necesidad de alentar la movilidad de profesores y alumnos y de establecer una equivalencia, no sólo en materia de títulos, sino también de estatutos, de exámenes, y de concesión de Becas, e insta a los rectores firmantes a trabajar para que los Estados y organismos públicos implicados colaboren en el cumplimiento de las metas acordadas.

		Actualmente, la Charta Magna Universitatum ha sido suscrita por 660 universidades de 78 países\footnote{Puede consultarse la lista de universidades en http://www.magna-charta.org/magna\_universities.html}. 

		\subsubsection{\bfseries \itshape Declaración de La Sorbona}
		Posteriormente a la ratificación de la Charta Magna Universitatum, el 25 de Mayo de 1998 los ministros de educación de Francia, Alemania, Italia y Reino Unido se	reúnen en La Sorbona (París) y dictan una declaración conjunta que venía a dar un sustento político a la declaración recogida en la Charta Magna Universitatum.

		En la Declaración de La Sorbona \cite{UNISOR:DS-98}, realizada en el septuacentésimo trigésimo primer aniversario de su fundación, los ministros de educación de los países arriba mencionados vienen a reconocer que Europa es una realidad supranacional, con un gran potencial humano gracias a los siglos de tradición universitaria, y promueve la creación de un marco eurpeo donde las entidades nacionales y los intereses comunes puedan relacionarse y reforzarse para el beneficio de Europa. Reconoce la pérdida de movilidad de los estudiantes y el empobrecimiento que eso causa a la sociedad, por lo que aboga por una vuelta al modelo clásico, donde el alumno pueda enriquecerse con estudios realizados en otras realidades sociales.

		Establece como sistema de facto el aceptado actualmente (dividido en dos ciclos, llamados aqui universitario y de posgrado donde se realizará una elección entre una titulación de master o una de doctorado más extensa.) y fija el sistema de créditos ECTS como el óptimo para lograr la comparabilidad y convalidación entre diferentes países. 

		\subsubsection{\bfseries \itshape Declaración de Bolonia}
		El 19 de Junio de 1999 se firma en la Universidad de Bolonia (detonante de todo el proceso con su Cartha Magna Universitatum) el documento que da nombre al proceso de convergencia hacia un Espacio Europeo de Educación Superior (Proceso de Bolonia). La declaración de Bolonia \cite{UNIBOL:DB-99} fué firmada por 29 ministros con competencias en educación superior. Supuso el espaldarazo definitivo a la creación del Espacio Europeo de Educación Superior. La Declaración de Bolonia fue redactada con la vista puesta en la Charta Magna Universitatum, pero sobre todo, en la Declaración de la Sorbona firmada el año anterior y que constituye el primer espaldarazo político a la creación de un área Europea de Educación Superior.

		En ella se insiste de nuevo en la realidad supranacional en que se ha ido convirtiendo Europa, y de la concienciación creciente de la sociedad de la necesidad de construir un Europa con una sólida base intelectual, cultural, social y científico-tecnológica. 

		La declaración de Bolonia fija como meta última para Europa el lograr establecer un Área Europea de Educación Superior, y promocionar el sistema Europeo de enseñanza superior en el resto del mundo. Para lograr alcanzar estas metas propuestas, se establecen seis objetivos a cumplir a medio plazo (antes de la primera década del siglo XXI) que son:

		\begin{itemize}
			\item Adopción de un sistema de titulaciones fácilmente comprendible y comparable, gracias a la creación del suplemento del diploma, de modo que se facilite la obtención de empleo dentro del marco Europeo y la competitividad de su sistema educativo superior.
			\item Creación de un sistema basado en dos ciclos, pregrado y grado, de modo que sólo se pueda acceder al segundo ciclo una vez se haya superado satisfactoriamente el primero, con un periodo mínimo de tres años. El diploma obtenido después del primer ciclo será reconocido en el mercado laboral como un nivel adecuado de preparación. El segundo ciclo conducirá a la obtención del título de master o doctorado. \todo{confirmado, en la declaración de bolonia hablan de pregrado y grado para referirse a lo que luego será grado y master}
			\item Establecimiento de un sistema de créditos para propiciar la movilidad del alumnado. Estos créditos además, se podrán obtener por la realización de actividades no lectivas, siempre en el modo en que esté estipulado por la universidad receptora.
			\item Promoción de una movilidad efectiva, venciendo las trabas existentes a la libre circulación, y prestando especial atención a:
			\begin{itemize}
				\item el acceso a estudios y otras oportunidades de formación y servicios relacionados de los alumnos.
				\item reconocimiento y valoración de los periodos de estancia de los profesores, investigadores y personal de administración en instituciones europeas de investigación, enseñanza y formación, todo ello sin prejuicio de sus derechos estatutarios.
			\end{itemize}
			\item Incremento de la cooperación Europea para asegurar la calidad de la enseñanza, para lo cual se deberán desarrollar criterios y metodologías comparables.
			\item Adecuación de las dimensiones Europeas de educación superior que tengan como objetivo el desarrollo curricular, cooperación interinstitucional, mejora de los esquemas de movilidad y de los programas integrados de estudio, formación e investigación.
		\end{itemize}

		Termina la declaración de Bolonia indicando que la creación del Área Europea de Educación Superior se logrará respetando las singularidades de cada país, la diversidad de lenguas, culturas, los diferentes sistemas de educación nacional y la autonomía de las Universidades, y establece un calendario de reuniones para realizar un seguimiento del proceso de implantación del Área Europea de Educacion Superior. \todo{Confirmado, en Bolonia, aún se habla de área europea de educación superior}

	\subsection{El EEES en la actualidad}
    
	Los acuerdos de Bolonia son de una gran complejidad. Hemos de tener en cuenta que en la filosofía del EEES no está la homogeneizción de la educación superior, sino el establecimiento de un método de comparación y compatibilización entre los diversos sistemas educativos, respetando la singularidad de cada uno de ellos. El desarrollo de una metodología que permita la comparabilidad y compatibilidad entre los diferentes sistemas educativos permitirá una mayor movilidad de estudiantes, profesores, investigadores y trabajadores, cumpliendo los objetivos pactados en la declaración de Bolonia. Esta mayor integración de los diferentes sistemas educativos europeos y la mayor movilidad de las personas que los componen, harán que Europa tenga una mano de obra mejor cualificada, lo que sin duda redundará en una mayor competitividad de la industria, ciencia y servicios europeos.
    
		\subsubsection{\bfseries \itshape Bolonia Follow-Up Group}
    
		Desde el año 1999 y cada dos años, se viene celebrando con regularidad una cumbre ministerial para hacer balance del progreso realizado en la implantanción del Área Europea de Educación Superior, y establecer metas a cumplir de cara a la celebración de la próxima cumbre. 

		El encargado de organizar estas cumbres es el BFUG (Bolonia Follow-Up Group). El BFUG\todo{incluir nota al pie con la dirección del organismo} es el encargado de organizar las cumbres ministeriales y de elaborar el plan de trabajo, calendario de seminarios y otras actividades de interés para todos los participantes en el proceso. El BFUG está presidido por el país a cargo de la Presidencia de turno de la Unión Europea y estacompuesto por los ministerios afectados por el Proceso Bolonia de los 47 países integrantes, la Comisión Europea y varias organizaciones europeas, como la Asociación de Universidades Europeas (EUA\footnote{http://www.eua.be/}) o el Centro Europeo de la UNESCO para la Educación Superior (UNESCO-CEPES\footnote{http://www.cepes.ro/}), estás últimas como miembros únicamente consultivos.

		Las cumbres ministeriales celebradas hasta la fecha han sido Praga 2001\footnote{http://www.bologna.msmt.cz/PragueSummit/index.html}, Berlin
      2003\footnote{http://www.bologna-berlin2003.de/}, Bergen 2005\footnote{http://www.bologna-bergen2005.no/}, Londres 2007\footnote{http://www.bologna-bergen2005.no/} y Lovaina 2009\footnote{http://www.ond.vlaanderen.be/hogeronderwijs/bologna/} \todo{Comprueba que no haya habido cumbres posteriores en 2011 y 2013}. En esta última cumbre se fijaron las prioridades para el Área Europea de Educación Superior, a cumplir en el próximo decenio, que son:

		\begin{itemize}
			\item Acceso equitativo para todos los grupos sociales.
			\item Reconocimiento de las habilidades y competencias obtenidas fuera del marco puramente académico.
			\item Acceso al mercado laboral.
			\item Enseñanza centrada en el alumno.
			\item Educación, investigación e innovación.
			\item Cooperación internacional.
			\item Movilidad del alumnado y profesorado.
			\item Recogida de información, para un mejor seguimiento de la implantación de Bolonia.
			\item Herramientas transparentes para la comparación de titulaciones.
			\item Financiación de las Universidades.
		\end{itemize}
      
		También se solicita al BFUG que prepare un plan de actuación para poder avanzar en las prioriades marcardas, y se le pide de manera específica que:
		\begin{itemize}
			\item Defina unos indicadores para medir la movilidad de alumnos y profesores.
			\item Considere el modo de que se pueda lograr una mobilidad equilibrada, con un flujo total neutro, dentro del Área Europea de Educación Superior.
			\item Controle el desarrollo de mecanismos de transparencia para ser estudiados en la próxima conferencia de ministros que tendrá lugar en Bucarest en Abril de 2012. \todo{Comprueba e incluye información de esta conferencia ministerial}
			\item Cree una red que dé soporte a la expansión de Bolonia fuera del Área Europea de Educación Superior, haciendo un uso óptimo de las estructuras ya en funcionamiento.
			\item Siga desarrollando recomendaciones para el análisis de los distintos planes nacionales para el reconocimiento de créditos.
		\end{itemize}

		Con motivo del aniversario de la Conferencia de Bolonia, en el que se fijaban unos objetivos a cumplir antes de 2010, se celebró el pasado 11 y 12 de Marzo de 2010 un encuentro entre los países participantes en el Proceso de Bolonia para lanzar definitivamente el Área Europea de Educación Superior, comprometiéndose todos ellos a cumplir los objetivos marcados en Lovaina 2009 en la fecha prevista (2019). Este encuentro viene a plasmar el compromiso firme de los países participantes y las instituciones educativas contenidas en ellos con los principios acordados 11 años antes en la Universidad de Bolonia.
      
	\todo{Lo dicho, comprueba lo acontecido desde 2011, en concreto si ha habido cumbres ministeriales en el periodo 2011 - 2013 y las conclusiones de la conferencia de ministros de Bucarest en 2012}
      
		\subsubsection{\bfseries \itshape Antecedentes del ECTS}
      
		La herramienta que permitirá el cumplimiento de los objetivos fijados en Bolonia, y sobre la cual gira todo el EEES, es el sistema de créditos European Credit Transfer System.
		    
		Los créditos ECTS fueron establecidos\cite{ECTS-GUIA:2004} por el programa ERASMUS como una herramienta que permitía gestionar la movilidad académica de estudiantes y profesores acogidos al programa. Más tarde pasaron a utilizarse en el programa SOCRATES, y finalmente fueron adquiridos como unidad en el EEES, después de doce años de utilización dentro de los programas antes mencionados.
    
		Los créditos vigentes antes de la entrada en vigor del EEES eran una medida de las horas lectivas que cada asignatura tiene asignadas. Eran créditos útiles para medir horas "de aula", pero que de ningún modo tenían en cuenta el esfuerzo global que el alumno tenía que hacer para superar la asignatura. Existía por tanto la anomalía de asignaturas eminentemente prácticas, con más horas de trabajo fuera del aula que dentro, que tenían asignados menos créditos que otras asignaturas más teóricas, que requerían menos trabajo fuera del aula. Es decir, un alumno debía dedicar más esfuerzo (más horas de trabajo) a una asignatura que le reportaba menos créditos, que a otra cuyo aporte de créditos al global de la titulación era mayor y requeria menos horas de trabajo.
    
		Los créditos ECTS vienen a eliminar la problemática sobre la contabilidad de los créditos, unificando la unidad de medida del trabajo preciso para la obtención del título. Los créditos ECTS contabilizan la carga de trabajo necesaria para que el alumno supere con éxito una materia. Más adelante se hablará de los criterios para considerar si un alumno ha superado con éxito o no una asignatura.
    
		\subsubsection{\bfseries \itshape Sistema ECTS} 
		El sistema de créditos ECTS\cite{ECTS-GUIDE:2009} es la piedra angular sobre la que se sustenta el EEES, sirve como nexo de unión a todos los países integrados en el proceso de Bolonia, la mayoría de los cuales han incorporado los créditos ECTS dentro de sus respectivos sistemas educativos.
    
		Los créditos ECTS son una medida de la carga de trabajo que debe realizar el alumno para obtener las competencias precisas para superar con éxito una asignatura. La carga de trabajo antes mencionada incluye tanto actividades del aula (como clases magistrales, seminarios o exámenes...) como actividades fuera de ella (realización de proyectos, estudio solo o en compañía...).
    
		Se asignan 60 créditos ECTS a la carga de trabajo de todo un año lectivo, con la adquisición de las competencias asociadas. De modo general, un año lectivo suele constar de ente 1.500 y 1.800 horas de trabajo, lo que nos arroja un valor para el crédito de entre 25 y 30 horas de trabajo.
    
		Los créditos se asignan para la titulación completa y luego se distribuyen proporcionalmente entre los componentes de la titulación en función del peso que cada uno tiene en la titulación. Los créditos se otorgan al alumno al finalizar las actividades formativas marcadas por el programa de estudios y al comprobar el aprovechamiento de las actividades, es decir, si ha logrado adquirir o no las competencias requeridas para la superación de ese hito.
    
		Es por tanto posible que, si un alumno ya ha adquirido con anterioridad ciertas competencias, los créditos asociados a la adquisición de dichas competencias sean convalidados al alumno, una vez que se haya comprobado mediante el oportuno reconocimiento la adquisición de dichas competencias.
    
		Además, los créditos pueden ser transferidos de un programa a otro, ya sea este ofrecido por la misma insitutción o por otra diferente. Para que la transferencia de créditos pueda llevarse a cabo, es preciso que la institución receptora de dichos créditos reconozca los créditos otorgados y las competencias reconocidas por la institución oferente. 
    
		A continuación veremos en detalle las líneas maestras del ECTS:
		\begin{description}
    			\item[El ECTS es un sistema de créditos centrado en el aprendizaje.]Su filosofía es una ayuda a las instituciones para el cambio de los sistemas de aprendizaje, desde uno centrado en el profesor, donde los créditos antiguos miden horas lectivas, hacia el centrado en el alumno propuesto por Bolonia, donde el crédito ECTS mide carga de trabajo para el alumno.
    	
    			Además, los créditos ECTS en conjunción con el sistema de adquisición de competencias permite establecer una correlación entre la oferta educativa y el mercado laboral, prolongar de manera contínua la educación y adquisición de competencias al flexibilizar los programas de estudios y facilitar el reconocimiento de los créditos y competencias ya adquiridos, y permite la movilidad entre instituciones de enseñanza, países y contextos de aprendizaje al facilitar una unidad de equivalencia entre todos ellos. 
    	 
		    	Los resultados del aprendizaje describen lo que se espera que los alumnos sepan, comprendan y sean capaces de realizar, una vez finalizado con éxito el itinerario formativo. La definición de resultados del aprendizaje clarifican los objetivos de los programas de estudio y permiten de este modo que estudiantes, profesores y demás partes implicadas en el proceso de aprendizaje comprendan con mayor profundidad los programas de estudio, asi como facilitar la comparación entre diferentes cualificiones o comprender de manera más fácil la aptitudes logradas en una titulación.
    	
    			La identificación de los resultados del aprendizaje en un plan de estudios es crítica, ya que de ella dependerá e buena medida el cálculo de la carga de trabajo, y por tanto, los créditos asignados. Una vez que los responsables del desarrollo de los resultados del aprendizaje de una titulación han establecido el perfil y los resultados esperados de dicha titulación (y de cada uno de sus componentes o módulos), el uso de créditos ECTS les permite estimar de manera más realista la carga de trabajo necesaria, y a elegir los métodos docentes y las diferentes actividades formativas más adecuadas.
    	
    			Los resultados del aprendizaje de un componente del programa educativo (asignatura, módulo, etc), debe ir acompañado de un criterio de evaluación claro y adecuado, ya que los créditos únicamente se otorgarán al alumno una vez se haya comprobado que éste ha adquirido los conocimientos y competencias definidos en el resultado del aprendizaje. Los resultados del aprendizaje permiten evaluar conocimientos y destrezas adquiridos en otros entornos diferentes a la educación superior normal, asignarles créditos y que de este modo sean reconocidos a la hora de la consecución de un título formativo.
    	
				%\todo{Cómo queda mejor, con paragraph o sin él?} SIN EL
				%\paragraph{Resultados del aprendizaje y competencias.}
				%\label{sec:ResultadosDelAprendizajeYCompetencias}
			En Europa se utilizan diversas definiciones de \textit{resultados del aprendizaje} y \textit{competencias}. No obstante todos ellos están relacionados de una u otra manera con aquello que el estudiante va a conocer, comprender y ser capaz de hacer al final del ciclo educativo. El uso de estos dos términos forma parte troncal del nuevo paradigma de la educación según Bolonia, situando al estudiante en el centro de todo el proceso educativo.
					
			En el Marco de Cualificaciones para el EEES\todo{Buscar una cita}, los resultados del aprendizaje y las competencias se consideran productos globales del aprendizaje. Son declaraciones genéricas de expectativas de niveles de consecución de competencias y habilidades relacionados con el ciclo de Bolonia. En este marco el término \textit{competencia} adquiere un sentido amplio, lo que permite matizar entre diferentes niveles de adquisición de habilidades y destrezas.
					
			El Marco Europeo de Cualificaciones para el Aprendizaje Permanente\todo{Busca una cita} diferencia entre conocimiento, destrezas y compentencias. El término \textit{competencia} se define aqui como "`la capacidad demostrada de utilizar conocimientos, destrezas y habilidades personales, sociales y metodológicas en situaciones de estudio o trabajo, y tanto en el ámbito personal como en el profesional."' En este caso el término \textit{competencia} se entiende como la capacidad para transferir los conocimientos a la práctica. 
					
			El Proyecto Tuning \todo{Busca una cita}distingue entre resultados del aprendizaje y competencias para distinguir los roles de profesores y estudiantes. En el contexto de Tuning, las competencias representan una combinación de conocimientos, comprensión, destrezas, habilidades, y actitudes, y se establece la distinción entre compentencias genéricas y las específicas de una disciplina. Desarrollar todas estas competencias es el objetivo del proceso de aprendizaje y de un programa educativo. El profesorado identifica los resultados del aprendizaje, que expresan el nivel compentencia adquirido por el alumno.

			\item[ECTS, niveles académicos y descriptores de los niveles.]Los marcos de cualificaciones europeos se basan en descriptores consensuados de los niveles academicos. Estos descriptores llevan asociados resutlados del aprendizaje y créditos. El Marco de Bolonia ha acordado descriptores de ciclo con resultados del aprendizaje y un rango de créditos. 
				
			Estos descriptores permiten conocer las expectativas habituales de logros y habilidades asociados con las cualificaciones que representan el final de cada uno de los ciclos de Bolonia. Estos descriptores no representan un umbral o unos requisitos mínimos, y tan solo pretenden identificar la naturaleza de la cualificación en su conjunto. 
				
			Los dos primeros ciclos de Bolonia están relacionados con los siguientes rangos de créditos ECTS. Las cualificaciones de primer ciclo incluyen entre 180 y 240 créditos ECTS. Las cualificaciones de segundo ciclo incluyen entre 90 y 120 créditos ECTS. 60 créditos ECTS componen la carga de trabajo de un año académico normal a tiempo completo, dentro de un programa de aprendizaje formal. Esta norma se aplica, independientemente del nivel, a todas las cualificaciones de educación superior.
				
			Los marcos de cualificaciones pueden contener niveles intermedios dentro de los ciclos de Bolonia, lo que permite a las institutciones organizar una titulación y regular la progresión del alumno a lo largo de ésta. Los créditos siempre se describen en función del nivel en que se otorgan y según los resultados del aprendizaje de un programa o componente del mismo. Sólo los créditos otorgados en un nivel pueden acumularse para la obtención del título 
				
			\item[Creditos ECTS y carga de trabajo.]La carga de trabajo indica las horas de trabajo que el alumno debe invertir en las diferentes actividades formativas necesarias para alcanzar los resultados del aprendizaje esperados.
				
			Previamente a calcular la carga del trabajo vinculada a un programa académico deben definirse los resultados del aprendizaje, en función de los cuales se elegirán las actividades formativas adecuadas y de este modo calcular la carga de trabajo necesaria para llevarals a cabo. 
				
			El cálculo de las horas de trabajo no debe basarse únicamente en las horas que el personal docente y el alumno comparten, sino que debe abarcar todas las horas que el alumno dedica a las diferentes actividades formativas, incluidas aquellas que el alumno desarrolla por su propia cuenta o en colaboración con otros alumnos, incluso aquellas destinadas a la preparación y el desarrollo de las pruebas de evaluación.
				
		\end{description}
 
\section{Motivación}

	Como vemos, el proceso de Bolonia introduce cambios sustanciales en la educación superior europea. Estos cambios tan profundos en el paradigma de la educación conllevan la introducción de muchos conceptos, algunos de ellos de nueva creación. Estos conceptos no son siempre conocidos para el personal que debe desarrollar los planes de estudio, e incluso en el caso de que el personal docente conozca el significado de dichos términos, pueden existir diversos matices en cuanto a su interpretación, lo que en la práctica conduce a malentendidos y equívocos. 
  	
		\subsection{Crítica al sistema de hojas de cálculo}
	  	%disgregación de la información, redundancia, incoherencia, etc.)
		Con la llegada del nuevo plan de estudios, fue necesario redistribuir las horas docentes de cada asignatura, dividiendo las asignaturas en créditos ECTS donde antes estaban divididas en créditos y pasando a adoptar los criterios de Bolonia. Con la finalidad de facilitar esta transción hacia una visión "`alumno"' del grado, se crearon unas hojas de cálculo destinadas a aglutinar toda la información relativa a las asignaturas de un departamento y a distribuir las horas lectivas de cada asignatura. 
      
		Cada hoja de cálculo (existen copias de las mismas en el anexo del presente documento)\todo{inclúyelas, no se te olvide}  contiene información acerca de una única asignatura, siendo cada una asignatura algo "`estanco"' pues no contenían información y relaciones con otras asignaturas. Además, no existe el concepto de materia y los concepctos de actividad formativa y método docente aparecen entremezclados y pueden dar lugar a errores. Resulta imposible para un alumno conocer qué compentencias va a desarrollar con el curso de la asignatura, las horas de dedicación que le exigirá, qué asignaturas va a poder cursar tras haber cumplido con los requisitos. Y el profesor no puede planificar su agenda para la docencia de varias asignaturas, esta obligado a utilizar herramientas auxiliares para la programación de las materias, no conoce las horas que el alumno debe dedicar a cada actividad docente para planificar la asignatura, qué métodos docentes son válidos para cada asignatura\ldots Vamos a verlo en detalle:
      
		\begin{description}
			\item[Hoja 1:] Consta de una pequeña nota con la definición de los diferentes métodos docentes.

			\item[Hoja 2:] Consta de una pequeña reseña con la definición de las diferentes	actividades formativas.

			\item[Hoja 3:] En esta hoja, denominada "`Plantilla Alumnos"' se recogen diferentes estadísticas sobre los alumnos, los años que tardan en finalizar los estudios y previsiones sobre el rendimiento de los alumnos.

			\item[Hoja 4:] En esta hoja, llamada "`Plantilla Prof-Dept"', se dividen los grupos de alumnos en cuatro clases en función del núemro de alumnos que componen cada grupo (A, B, C ó D) dependiendo del número de alumnos en cada grupo. Luego, en función del número de profesores presente en el departamento y su disponibilidad, la hoja calcula el total de horas disponibles para la docencia.

			\item[Hoja 5:] En la quinta hoja del libro de excel, llamada "`Plantilla ACT y MET"', se recoge, en una misma hoja de cálculo, los datos de la asignatura (nombre, número de créditos, etc), prerequisitos, número de horas de cada actividad formativa, métodos docentes aplicados, número de horas dedicadas a la preparación de la evaluación de cada métodos evaluador, y capacidades adquiridas por el alumno tras cursar esta asignatura. Al final, la hoja recoge el total de horas utilizadas por el alumno y las transforma en créditos ECTS. Adicionalmente, debajo de este cuadro, aparecen repartidas las horas de docencia y las horas destinadas a evaluación por el departamento. De este modo se intentan distribuir las horas docentes de la asignatura entre todas las actividades formativas, teniendo presentes las competencias que se espera que el alumno adquiera cursándola.

			\item[Hoja 6:] Es una versión de la hoja anterior donde además aparecen asignados los profesores como recursos, en un intento de lograr que las horas de docencia comprometidas sean cubiertas por los profesores disponibles.

			\item[Hoja 7:] En esta séptima hoja, llamada "`Plantilla C\_ESPECIFICAS"' vienen recogidas todas las competencias específicas del plan de estudios, el número de horas dedicadas a cada actividad formativa, las actividades formativas que permiten al alumno adquirir cada capacidad y el nivel adquirido por el alumno a la finalización de la asignatura.

			\item[Hoja 8:] En esta última hoja del fichero, "`Resultado C\_Generales"', se	recoge en una tabla todas las competencias generales que se pueden adquirir en el curso de la titulación, junto con el número de horas dedicadas a cada actividad formativa para la adquisición de cada competencia. Una última columna	nos indicará el total de horas dedicadas a la adquisición de cada competencia.
		\end{description}

		Como primer apunte al método empleado, podríamos subrayar el uso de una herramienta como es la hoja de cálculo para un fín que no es el propio. Como consecuencia, tenemos tablas ineficientes, con muchos datos, campos con texto mezclados con números, y que a primera vista resultan muy poco claras. Este sistema de hojas de cálculo es claramente ineficaz, y su principal problema es la ausencia de un criterio único a la hora de otorgar significado a los diferentes conceptos. Toda la información queda embarullada y mezclada, y resulta muy difícil rellenar las hojas para una única asignatura, de modo que rellenar los datos completos de toda una materia o incluso un grado resulta una tarea excesivamente dificultosa.
     
		Por otra parte, no parece muy útil el conocer que un grupo sea de clase A, B, C o D, al menos de cara al profesor. Los grupos se establecen en función del número de alumnos que cursan cada asignatura, y por tanto, la planificación de ésta es independiente del número de alumnos que la cursen. El nuevo paradigma de la educación superior centra su atención en los alumnos. Los créditos miden las horas de trabajo de los alumnos, en ningún caso el trabajo de los profesores (no por ello despreciable). Dicho de otro modo, los créditos precisos (las horas de trabajo que un alumno precisa dedicar) para superar la asignatura son los mismos, haya pocos alumnos o muchos alumnos, y el hecho de utilizar diferentes métodos docentes o realizar unas u otras actividades formativas tiene más que ver con el hecho de que sean aplicables a la docencia de esa asignatura que a la oferta de infraestructura del centro docente (aunque en la práctica sea éste una limmitación importante a la hora de impartir las asignaturas).
      
		Ya se ha visto como los conceptos de método docente y el de actividad formativa quedan difuminados y se entremezclan. Una actividad formativa es la actividad que profesor y alumno han de realizar a lo largo de un curso, que busca un propósito concreto. Por tanto es válido decir que para una misma materia concurren a lo largo del curso docente varias actividades formativas, ponderadas en función de los objetivos propuestos en el plan de estudios. Por el contrario, un método docente es un conjunto de formas, procedimientos, técnicas, etc, de enseñanza y aprendizaje. Por tanto, y a pesar de que existen actividades formativas y métodos docentes que son mutuamente excluyentes, se puede afimar que por regla general pueden conbinarse entre sí a criterio del docente, siendo totalmente compatibles el uso de diversos métodos docentes con las diferentes actividades formativas. Por ello, tiene sentido de hablar de horas dedicadas a una determinada actividad formativa, pero no hablar de horas dedicadas a un método docente. 
      
		Por otro lado, en la hoja de cálculo no se mencionan los diferentes niveles de adquisición de competencias\todo{Ojo, nosotros en la ontología tampoco hablamos de niveles de adquisición de competencias, pero es que en el documento remitido a ANECA no figura el grado de maestría de una competencia una vez que el alumno ha superado con éxito una asignatura.}, sino que se habla de horas dedicadas a esa competencia. No es esta información la figura en el documento remitido al ANECA, que ha servido de guía para la comprensión del plan de estudios. Salvo que se establezca de forma explcíta la correspondencia entre los tres niveles (alto, medio y bajo) de adquisición de compentecias indicados en el documento y las horas de dedicación correspondientes a cada nivel, no tiene sentido establecer escalas que luego no podrán ser utilizadas en el documento final del grado. Su única utilidad es la de unidad de medida interna a cada departamento docente, con lo cual tenemos además el riesgo de que cada departamento pueda utilizar diferentes gradaciones para la adquisición de competencias.
      
		En resumen, el sistema de hojas de cálculo empleado adolece de:
		\begin{itemize}
			\item Falta de precisión en los conceptos.
			
			\item Inexistencia de límites en la asignación de horas de trabajo a las asignaturas, quedando supeditada la corrección de la asignación de horas al buen hacer de la persona que rellena hoja.
			
			\item Falta de control en la adquisición de competencias, quedando de nuevo supeditado al buen hacer de la persona que rellene las hojas de cálculo.

			\item Exceso de información en cada hoja de cálculo. Por ejemplo, en las competencias específicas y generales se muestran las de todo el plan de estudios, en lugar de tan solo las competencias que deban ser adquiridas al cursar dicha asignatura.

			\item Inexistencia de relación entras las asignaturas y la materia en que se engloban.

			\item Estanqueidad en el diseño del plan. La asignatura cursada forma parte de una materia y esa materia de un plan de estudios. Esa relación debe estar plasmada, dado que no son conceptos aislados, si no que están muy estrechamente ligadas, por ejemplo, por la existencia de pre-requisitos (realmente recomendaciones del centro docente) en el acceso a ciertas asignaturas.
	      \end{itemize}

		Además de estas carencias, poniendo la vista en un medio-largo plazo, resulta a priori muy complicado el cambio de alguna especificación del plan utilizando las hojas de cálculo. Tienen una cohesión muy pequeña entre ellas por lo que cualquier modificación del plan de estudios o del marco normativo resulta muy dificil de trasladar correctamente a las hojas de cálculo. Es preciso construir la plicación de manera que en la ampliación o modificación del sistema resulta una tarea accesible.
      
		Además, la automatización de tareas resulta casi imposible utilizando hojas de cálculo. Resulta incongruente pensar que, hoy en día, dependamos del tratamiento manual de datos para la correcta definición de cualquier sistema o metodología, habiendo como hay herramientas que nos facilitan su diseño y desarrollo.

	\subsection{Ontologías}
  	
	Ocurre a menudo que cuando varias personas materializan en la mente un objeto, cada una de ellas lo hace con diferentes cualidades o distinto nivel de detalle. Cuando pensamos en un \textit{vehículo}, algunos imaginarán una moto, otros un coche, un ciclomotor, etc. Todas las representaciones tienen una estructura básica en común (un objeto que permite desplazarse a personas) que es aquello que realmente define el concepto de \textit{vehículo}, aunque como hemos visto cada uno imagina ese vehículo de acuerdo a sus experiencias o preferencias personales. 
  	
  	Y aunque todos imaginásemos el mismo tipo de vehículo, como un \textit{coche}, esta definición puede ser más concreta o abstracta en función de las necesidades del universo que estamos describiendo. Para un recaudador de impuestos municipales, un \textit{coche} se reduce a un vehíclo a motor, con una matrícula, unos caballos fiscales y un obligado tributario. Para la persona que va a comprar un \textit{coche} a un concesionario le importará saber si tiene tres o cinco puertas, consumo, prestaciones, habitabilidad, equipamiento, etc. Y para el dueño de la concesión, un \textit{coche} se reduce a un apunte contable en la cuenta del debe (una vez descontados los costes de adquisición y rotación del \textit{coche}).
  	
	Es importante por ello acometer una unificación de los conocimientos antes de poner en marcha el diseño del plan de cara a evitar errores de diseño derivados de un error conceptual. Las ontologías son herramientas utilizadas para modelar el universo (una parte al menos) de modo que todos los actores implicados en el desarrollo del plan tengan una visión unificada e inequívoca del mismo.
  	 
		\subsubsection{Definición}
    
		Una ontología\todo{añade una referencia} no es un vocabulario ni un diccionario donde figuran las definiciones de los conceptos utilizados. Una ontología es un mapa donde conceptos y significados se entrecruzan. Se trata de una forma de representación del conocimiento que permite tener un entendimiento común y compartido de un dominio, de modo que diferentes personas o sistemas puedan compartir una misma visión de ese dominio. 

		Existen varias definiciones formales de Ontología. Varios autores refieren la definición tal y como a la facilita Tom Gruber\cite{GRUBER:93}. Según esa definición, una ontología es una especificación de una conceptualización. Otra definición más concreta es la ofrecida por Weigand\cite{WEIGAND:97}, según el cual una ontología es una base de datos que describe los conceptos del mundo o algún subdominio, algunas de sus propiedades, y como se relacionan cada uno de los conceptos. Para un sistema basado en el conocimiento, podemos asumir que sólo existe aquello que podemos representar, y que todo aquello que no pertenece al dominio de la ontología, no existe. 

		El uso de ontologías implica por tanto la definición de un vocabulario y reglas gramaticales que relacionen los vocablos. Estas reglas gramaticales nos permitirán realizar preguntas a la ontología cuyas respuestas deberán ser, forzosamente, coherentes con las definiciones y constantes de la ontología. Todas estas propiedades de las ontologías nos permitirán:

		\begin{itemize}
			\item Intercambiar datos entre diferentes sistemas.

			\item Crear servicios de consulta.
			
			\item Crear bases de conocimiento reusables.

			\item Ofrecer servicios para facilitar la interoperabilidad entre diversos sistemas y bases de datos.

		\end{itemize}
      
		Todas estas propiedades se pueden resumir diciendo que el uso de ontologías nos permitirá especificar una representación del modelo de datos a un nivel superior al del diseño de bases de datos específicas, lo que permitirá la exportación, traducción, consulta y unificación de la información a través de sistemas y servicios desarrollados de manera independiente.
      
		La palabra ontología se utiliza a menudo para hablar de métodos de modelado del universo estructurados en mayor o menor medida, como puedan ser las taxonomias\todo{referencia}, los tesauros\todo{referencia} o las ontologías. Las ontologías constan de clases (conceptos presentes en los diversos dominios de interés), relaciones entre dichas clases y las propiedades o atributos que tienen esos conceptos.
      
		De modo general, las ontologías se expresan en lenguajes basados en conceptos lógicos, de modo que se puedan distinguir de manera inequívoca las diferentes clases, propiedades y relaciones. Incluso algunas herramientas automáticas pueden realizar razonamiento autónomo sobre esas ontologías, como la búsqueda semántica y conceptual, agentes software, ayuda a la toma de decisiones, entendimiento del lenguaje natural, gestión del conocimiento, etcétera.
      
		Las ontologías son críticas para aplicaciones que deben gestionar información para diversas comunidades. El lenguaje XML\todo{referencia} es suficiente para el intercambio de información, pero todas las partes implicadas deben, con anterioridad al intercambio, haber llegado a un acuerdo sobre la semántica. Es más, la ausencia de dicha semántica es lo que impide que las aplicaciones puedan continuar con su trabajo, cuando cambia el vocabulario XML. El lenguaje RDF\todo{referencia} en cambio sí que permite la asociación de significado semántico a ciertos "`identificadores"', de modo que en RDF nos podemos encontrar con clases que poseen subclases y superclases, así como con propiedades, subpropiedades, dominios y rangos. Pero sin embargo, RDF carece de una semántica más rica y compleja que permita la definición de expresiones complejas, como clases disjuntas por poner un ejemplo.
      
		\subsubsection{Metas}
		Como regla general, las ontologías vienen a cubrir determinadas necesidades, para las cuales o bien son la única herramienta válida, o bien resultan ser la herramienta más adecuada. De cualquier modo, hay que aclarar que se trata de una mera enumeración de situaciones o casos de uso y no recomendaciones, y que no es preciso que se den todas y cada una de ellas para justificar el uso de ontologías.
      
		\begin {enumerate}
		
			\item El intercambio de información requiere para que sea efectivo del acuerdo previo, entre las partes que realizan dicho intercambio, sobre las definiciones de los identificadores utilizados. Las ontologías proveen de identificadores estándarizados y descripciones formales de esos identificadores, de modo que todos los usuarios de esa ontología acuerdan de modo implícito utilizar los mismos identificadores, y lo que es más importante, con el mismo significado. 
      	
			A menudo no es suficiente con usar ontologías compartidas. Pudiera ser que un usuario encontrase que una ontología existente cubre el 80\% de sus necesidades, pero que el 20\% restante no estuviese definido. Este usuario no tendría que comenzar a diseñar una ontología desde cero, sino que podría reutilizar las ya existentes y extenderlas hasta cubrir el 100\% de sus necesidades añadiendo identificadores y definiciones propias.
      	
      			Es por ello que las ontologías deberían estar a disposición de terceros de modo que los usuarios puedan completar las ya existentes para cubrir nuevos significados.
      	
      			\item El mundo cambia constantemente, y por ello sería desable que las ontologías también lo hiciesen al mismo compás. Las ontologías deberían ser capaces de evolucionar, dado que se pueden encontrar errores en versiones anteriores, o bien aparecer nueva terminología o una nueva forma de modelar el dominio. 
      	
      			Por ello es importante que cada versión de una ontología posea un número de versión. Es más, también se debería especificar qué versiones son o no compatibles, entre sí, no para eliminar las incompatibles, sino para distinguir unas de otras. Dado que es posible que en una revisión posterior se cambie el significado de un identificador sin modificar su descripción formal, es preciso que el autor de la revisión indique dichos cambios de manera explícita.
      	
      			Pero se ha de tener en cuenta que la evolución de una ontología es distinto de la extensión de ontologías, donde la ontología original no se ve modificada por su extensión.
      	
     			\item A pesar de que las ontologías compartidas permiten cierto grado de interoperabilidad entre dominios, existen casos donde es posible modelar de vairas formas el mismo universo, debido a diferentes perspectivas. Por ello, para permitir que los sistemas informáticos puedan procesar información proviniente de ontologías heterogéneas las ontologías deben soportar el uso de primitivas que permitan mapear conceptos hacia otras ontologías.
     	
     			Esto permite transformar, o si no al menos transportar conceptos desde unas hacia otras ontologías, creando de este modo una red de ontologías.
     	
     			\item Podemos encontrar, entre las distintas ontologías que modelen un mismo universo y aunque a priori puedan ser compatibles, inconsistencias que hagan dichas ontologías incompatibles. Es importante que las inconsistencias puedan ser detectadas de manera automática, de modo que se prevenga la corrupción de datos consistentes o la combinación de datos incompatibles.
     		 
		\end{enumerate}
      
	\subsection{Unificar conocimiento en un marco único y formal}
	
	El World Wide Web Consortium\todo{referencia} es un organización internacional, en el que participan organizaciones miembro, persoanl a tiempo completo y público en general con el fin de desarrollar estándares web que permitan sacar el máximo potencial de la web. 
		
		\subsubsection{¿Porqué desarrollar OWL?}
		
		La Web Semántica\todo{referencia} permite dotar de significado a la información, permitiendo que las computadoras puedan más fácilmente procesar e integrar la información presente en la web. Se hizo necesario entonces diseñar un lenguaje para la web ontológica que pudiese complir con los objetivos arriba mencionados fijados por el W3C. Existen varios lenguajes relacionados con la web semántica:
		
		\begin{itemize}
			\item[XML]\todo{referencia} Proporciona una sintaxis para documentos estructurados, pero no aporta ningúna semántica a esos documentos.
			\item[XML Schema]\todo{referencia} Restringe la estructura de los documentos XML y añade la posibilidad de definir tipos de datos.

			\item[RDF]\todo{referencia} Es un lenguaje para el modelado de datos con objetos (llamados recursos en el lenguaje) y relaciones entre ellos, creando una semántica para esos modelos y facilitando su representación en sintaxis XML.

			\item[RDF Schema]\todo{referencia} Es un vocabulario para la descripción de propiedades y clases de recursos RDF que incorpora una semántica para la definición de jerarquías de propiedades y clases.

			\item[OWL]\todo{referencia} Añade vocabulario para la descripción de clases y propiedades como pueden ser relaciones entre clases, cardinalidad, igualdad, más tipos de relaciones, propiedades de esas relaciones y clases enumeradas. 
		\end{itemize}
	
		\subsubsection{Sublenguajes de OWL}
		Existen tres sublenguajes de OWL, cada uno más expresivo que el anterior, diseñados para ser utilizados de manera específica por implementadores y usuarios, de modo que éstos puedan disponer de un lenguaje adaptado a sus necesidades. 
	
		\begin{enumerate}
			
			\item \textit{OWL Lite}\todo{referencia} está diseñado para usuarios que principalmente precisen una jerarquía de clases y constantes muy simples. Por poner un ejemplo, \textit{OWL Lite} permite el uso de cardinalidad pero sólo con valor 0 ó 1. \textit{OWL Lite} precisa de herramientas más sencillas que otros sublenguajes de OWL más expresivos y es además una manera rápida de de implementar y compartir tesauros y otras taxonomias.
		
			\item \textit{OWL DL}\todo{referencia} permite obtener la máxima expresividad del lenguaje a la par que completitud y decidibilidad sobre ese lenguaje. Incluye todos los constructores de OWL pero sólo permite que sean utilizados bajo ciertas condiciones, 
	
			\item \textit{OWL Full}\todo{referencia} está pensado para aquellos usuarios que desean trabajar con la máxima expresividad y libertad sintáctica, pero sin ninguna garantía de completitud ni de decidibilidad. 

		\end{enumerate}
		
		Cada uno de estos sublenguajes es una extensión de su predecesor, y por tanto:

		\begin{itemize}

			\item Todas las ontologías OWL Lite válidas son ontologías OWL DL válidas.

			\item Todas las ontologías OWL DL válidas son ontologías OWL Full válidas.

			\item Todas las conclusiones válidas sobre una ontología OWL Lite son conclusiones OWL DL válidas.

			\item Todas las conclusiones válidas sobre una ontología OWL DL son conclusiones OWL Full válidas.

		\end{itemize}
		
		\subsubsection{Elección de un sublenguaje OWL}

		A la hora de escoger uno u otro sublenguaje OWL, se ha de tener en cuenta las necesidades que va a tener que satisfacer. La elección entre OWL Lite y OWL DL dependerá de si OWL Lite es capaz de satisfacer la expresividad requerida o no. La elección entre OWL DL y OWL Full dependerá de que el usuario requiera hacer uso de los meta-modelos (posible con la sintaxis de RDF Schema) o no, y de la necesidad de utilizar razonadores sobre la ontología, dado que por el momento no existen razonadores para ontologías OWL Full. 
		
		EL sublenguaje OWL Full puede ser considerado como una extensión de RDF, mientas que OWL Lite y OWL DL pueden ser considerados como extensiones de un RDF restringido. Cualquier documento OWL (ya sea OWL Lite, OWL DL o OWL Full) es un documento RDF y cada documento RDF es un documento OWL Full, pero sólo algunos documentos RDF serán documentos OWL Lite o OWL DL válidos, debido a las restricciones de sintaxis que cada uno lleva aparejado. 
		
		Por ello, es preciso tomar precauciones cuando se desea convertir un documento RDF en uno OWL Lite o OWL DL, ya que el nuevo documento RDF debe cumplir con las restricciones propias de cada sublenguaje OWL. Para obtener más información acerca de las diferentes restricciones que cada sublenguaje impone, se recomienda ver el apéndice E de la referencia de OWL.
		
		
%      El uso de ontologías para la especificación del conocimiento nos permitirá
%      trabajar con una única representación del conocimiento, válida para todos los
%      actores involucrados en el proceso de educación. Esta visión única y global nos
%      permitirá que quienes quieran que sean las personas que precisen trabajar sobre
%      el conocimiento representado compartirán todas ellas una misma visión de
%      conjunto.
%      
%      De manera adicional, se logra que los conceptos utilizados en la descripción del
%      conocimiento estén definidos de manera precisa en la propia ontología, con lo
%      que se elimina la necesidad de acudir a fuentes externas, y por tanto se unifica
%      el significado de los conceptos introducidos.
%      De este modo, al unificar la representación del conocimiento y las significación
%      de sus conceptos, logramos crear un marco único y formal, que posibilita:
%      \begin{itemize}
%	\item la eliminación de ambigüedades, al trabajar todos los agentes sobre un
%	mismo marco de representación.
%	\item un rápido intercambio de ideas, ya que se elimina la necesidad de
%	"traducción".
%	\item dotar al modelo de conocimiento de una elevada adaptabilidad a cambios
%	futuros, pues las bases han sido creadas sin ambigüedades ni incertidumbres.
%      \end{itemize}
%
%    \subsection{Marco unificado}
%      %{Marco unificado para añadir (open world assumption OWA) y extraer información.}
%      Aqui tengo que hablar con Ángel, porque no sé muy bien qué poner aqui. Busca en
%      internet.

	\subsection{Procesos automatizados}
    	El hecho de que, como ya se ha comentado, las ontologías OWL DL y OWL Lite sean lenguajes completos y decidibles permite la utilización de herramientas automáticas, abriendo una amplio abanico de posibilidades en la utilización de dichas ontologías.
    	
    	Algunas herramientas automáticas proporcionan un interfaz de programación para el diseño e implementación de ontologías, exploradores de ontologías, conversores, razonadores, motores de búsqueda, etiquetadores, validadores, visores... Existen diversas páginas como \textit{http://www.w3.org/2001/sw/wiki/Tools}, \textit{http://semanticweb.org/wiki/Tools}, \textit{http://www.mkbergman.com/sweet-tools-simple-list/} o \textit{http://protegewiki.stanford.edu/wiki/Protege\_Plugin\_Library} donde hay repositorios y se pueden encontrar las herramientas agrupadas por categorías, lenguajes de programación \ldots A modo enunciativo, y para que el lector pueda tener una primera impresión de las capacidades del lenguaje, se enumeran algunas de las herramientas.
    	
	%\todo{Daniel, nombra aqui las páginas que has encontrado}
    	\begin{itemize}
		\item \textit{iQvoc} es una herramienta web de código abierto para manejar vocabularios, jerarquías, tesauros.
			
		\item \textit{ONKI} es un catálogo de ontologías, vocabularios y tesauros, fineses e internacionales, muy útil para poder construir y mantener contenido de la web semántica a un coste aceptable. 

		\item \textit{SMAD} es un catalizador para aplicaciones web para dispositivos móviles, que intenta integrar datos de la web semántica con servicios basados en la localización de esos dispositivos móviles.

		\item \textit{SKOS API} es un interfaz java, que incluye un parser y un editor con soporte para SKOS\todo{referencia} y en el que se pueden integrar razonadores como Pellet o FaCT++.

		\item \textit{TopBraid EVN} es una solución para el desarrollo y mantenimiento de vocabularios interconectados, de modo que todas las partes interesadas en un negocio o empresa pueden colaborar en defniir e interrelacionar vocabularios, taxonomias, tesauros y ontologías.

		\item \textit{Graphity} es una plataforma para la construcción de aplicaciones web, y permite explorar fuentes de datos externas, analizarlas e importar información proveniente de otras fuentes privadas de datos.

		\item \textit {Callicamchus} permite a los creadores de páginas web crear aplicaciones para la web semántica de manera rápida y sencilla.

		\item \textit{OntoVCS} es un conjunto de herramientas para la línea de comandos que permite obtener las diferencias de varias ontologías, así como unirlas y pueden integrarse en Mercurial\todo{referencia} o Git\todo{referencia} para el control de versiones.

		\item \textit{Pellint} es una herramienta para \textit{Pellet}\todo{referencia} que indica e incluso puede reparar diversas construcciones que se sabe que ocasionan problemas, a nivel de ontologías y axiomas.

		\item \textit{babel} es un servicio del proyecto SIMILE\todo{referencia} del MIT para la conversión desde o hacia diferentes formatos.

		\item \textit{Morph} es otro conversor entre varios formatos RDF.

		\item \textit{RDF On The Go} permite almacenar de manera persisitente estructuras RDF, así como procesar consultas, todo ello desde un sistema Android.
		
		\item \textit{TONES Ontology Repository} es un servicio web diseñado para hacer las funciones de almacén de ontologías de prueba para que los desarrolladores de aplicaciones puedar probar sus aplicaciones contra esas ontologías.

		\item \textit{Protégé} es un entorno de desarrollo de ontologías, del que se hablará más detenidamente en próximos capítulos.

		\item \textit{OWL Verbalizer} es una herramienta on-line que verbaliza las ontoloías OWL en Attempto Controlled English\todo{referencia}.

		\item \textit{eXperiment Design Automation} es una plataforma de desarrollo que posibilita a los científicos gestionar la información dentro y entre los laboratorios de investigación.

		\item \textit{Semantic Turkey} es una extensión para Firefox que permite a los usuarios construir ontologías con la información contenida en las páginas web visitadas. 

		\item \textit{FaCT++} es un razonador OWL DL, soporte para OWL y OWL 2. Es uno de los razonadores incluidos por defecto en Protégé.

		\item \textit{HermiT} es otro razonador, muy potente y útil para clasificar ontologías muy complejas y encontrar inconsistencias en ellas.

		\item \textit{RacerPro} es un razonador para la web semantica.
	
		\item \textit{Pellet} otro razonador que incluye como novedad la capacidad de realizar razonamiento incremental.
    	
	\end{itemize}
    	
    	Como ya se ha dicho anteriormente, el listado anterior es meramente informativo. Muchas de esas aplicaciones están en forma de plugins para entornos de desarrollo, o herramientas para la linea de comandos, o bien como aplicaciones java. 
    	
    	Mención aparte merecen Protégé y el resto de herramientas utilizadas para la realización del TFC, de los que se hablará más detenidamente en el siguiente capítulo.

      
\section{Objetivos}
  
  	Habiendo visto las carencias que presenta el sistema de las hojas de cálculo utilizado hasta el momento, es momento de crear un sistema de conocimiento que permita manejar de manera más fluida toda la información relativa a los planes de estudio. 
  	
  	Es por tanto la meta princpipal de este documento elaborar un marco único y formal que unifique el conocimiento, de modo que todos los conceptos sean  inequívocos y cualquier persona, aún a pesar de no estar familiarizada con la estructura de un plan de estudios pueda consultar, añadir y modificar información al sistema. Este marco único garantizará que toda la información incluida en el sistema sea coherente con el conocimiento modelado. Además, ello permitirá mejorar la presentación de la información, por lo que las relaciones entre asignaturas, materias y planes de estudios serán más claras, de modo que el usuario pueda ubicar en todo momento el lugar que ocupa en el universo modelado, comprendiendo de manera más amplia cómo encaja su labor y la información que maneja en el plan de estudios que estamos modelando. Así, crearemos un sistema mucho más flexible que permita, ante cambios normativos o cambios funcionales, adaptar más fácilmente el sistema de estudios a las nuevas situaciones cambiantes que se presenten. 
  	
	Con estos objetivos cumplidos, estaremos sentando las bases para tener un sistema estable y acotado, de modo que podamos, 
	\begin{itemize}
		\item Ampliar el modelo de conocimiento hacia otros planes de estudios, incluidos aquellos cursados en universidades extranjeras adheridas al marco europeo de educación superior.
		
		\item Crear o aplicar herramientas automáticas al conocimiento modelado, que faciliten el trabajo con la información contenida.

    \end{itemize}


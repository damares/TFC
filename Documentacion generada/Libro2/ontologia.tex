\chapter{Ontología de nuestro plan de grado}

En este capítulo veremos en más detalle el diseño de la ontología y las herramientas utilizadas para ello. Veremos también la aplicación de la ontología creada a un título específico (el grado en informática de la Universidad Politécnica).

\section{Herramienta utilizada:Protégé}
Protégé\cite{Stanford-Protege:WEB} es un framework para la edición de ontologías de código abierto. Las ontologías creadas con protégé se pueden exportar a muy diversos formatos, entre los cuales se incluyen RDF(S)\footnote{http://www.w3.org/RDF/}, OWL\footnote{http://www.w3.org/TR/2012/REC-owl2-overview-20121211/}, Manchester\footnote{http://www.w3.org/TR/owl2-manchester-syntax/} e incluso esquemas XML\footnote{http://www.w3.org/TR/2009/REC-xml-names-20091208/}.
Protégé está construido en Java y admite extensiones creadas por los usuarios por lo que constituye una excelente base para el desarrollo de aplicaciones o prototipos.
Adicionalmente, Protégé cuenta con amplica comunidad de usuarios (desarrolladores, docentes, estudiantes e instituciones gubernamentales y corporaciones privadas) que utilizan Protégé como base de conocimiento en áreas tan diversas como la biomedicina o la adquisicion de conocimiento.

\section{Introducción al documento de trabajo}
Para crear la ontología, se ha partido del documento \cite{FIUPM-MemoAneca}. Este documento es el remitido a ANECA\footnote{Agencia Nacional de Evaluación de Calidad y Acreditación - www.aneca.es} para su validación de acuerdo a las normas especificadas en el marco europeo. Aneca, como miembro de ENQA\footnote{European Association for Quality Assurance in Higher Education - www.enqa.eu}, EQAR\footnote{Eurpean Quality Assurance Register for Higher Education - www.eqar.eu} y INQAAHE\footnote{International Network for Quality Assurance Agencies in Higher Education - www.inqaahe.org} es el encargado de validar los planes de estudios y de certificar el cumplimiento de los requisitos marcados por Bolonia.

Con el propósito de poder comprender mejor la construcción de la ontología, vamos a comenzar haciendo un pequeño resumen de los puntos 3 y 5, referidos a los objetivos del título y a la planificación de las enseñanzas.
%\todo{HAY QUE HABLAR EN ESTA PARTE DE TODO LO QUE COMPONE LAS MATERIAS, ETC; DE ACUERDO CON LO DESCRITO EN LAS DATATYPE PROPERTIES:}

\begin{description}
	\item [Objetivos generales.]Los objetivos generales del título son la adquisición por parte del egresado, de unos niveles mínimos de adquisición de una serie de capacidades, competencias y destrezas generales. Los objetivos van a fijar las capacidades mínimas de todo alumno al finalizar los estudios.
 
	\item [Competencias generales y específicas.]Las competencias generales y específicas, llamadas ``objetivos'' en la denominación dada por el documento, son adquiridas por el alumno mediante la adquisición de diversas competencias, en distintos grados. La adquisición programada de estas competencias es lo que permitirá al alumno cumplir con los objetivos establecidos para el plan de estudios, y poder afrontar las diversas situaciones que en su vida laboral deberá afrontar.
 
	En general, están orientadas a la adquisición de aptitudes, actitudes y capacidades enfocadas hacia aspectos humanos, (como el trabajo en equipo, la motivación, o la actualización de conocimientos de manera autónoma, por citar algunos), que serán útiles al alumno en su incorporación al mundo laboral. Las competencias se diferencian entre ellas según sean específicas de un título (suelen estar orientadas más hacia aspectos prácticos de la titulación e incluyen la adquisición de metodologías y rutinas de trabajo y desarrollo eminentemente prácticas) o generales si lo que definen son competencias de carácter general, aplicables a varios títulos.
			 
	En cuanto al nivel de adquisición de cada competencia, se asumirá que el alumno alcanza un nivel de adquisición óptimo, de acuerdo a los objetivos propuestos, conforme va superando las distintas asignaturas. Dicho de otro modo, una asignatura únicamente puede superarse cuando el alumno ha adquirido ciertas competencias en un grado definido. Es tarea del evaluador, comprobar que el grado de adquisición de competencias por parte del alumno es el adecuado.

	\item [Materias.]El curso de las diferentes materias, debe asegurar la adquisición de todas las competencias, tanto específicas como generales, definidas en el perfil del título. El trabajo fin de carrera será considerado como una materia más. La materia optatividad, no consta de ninguna asignatura, para facilitar una rápida reacción ante cualquier cambio tecnológico, profesional o formativo que se produzca. Las materias están compuestas por asignaturas, de las cuales heredará una ubicación temporal (una materia esta ubicada en los trimestres en que estén ubicadas las asignaturas que la componen) y un carácter. Además, las materias tienen una duración en créditos ECTS que depende de la extensión de las asignaturas que la componen.
 
	Las materias se imparten mediante unas actividades formativas y según unos métodos docentes definidos en el título. Además, los docentes evaluarán la adquisición de competencias por parte de los alumnos mediante unos métodos de evaluación, también definidos en el perfil del título.
 
	\item [Asignaturas.]Debido a limitaciones del lenguaje, no se tendrán en cuenta limitaciones sobre la extensión de las asignaturas en créditos, limitaciones sobre la multiplicidad de los créditos, ni se hablará acerca de las horas reales a que equivale cada crédito ECTS. 
 
	Tampoco se controlará el número de asignaturas programadas por semestre, ni se determinará qué número de créditos están destinados a la adquisición de competencias transversales. Tampoco se controlará la correcta distribución de créditos por semestres, ni por asignaturas básicas, obligatorias u optativas. Todo esto queda fuera del alcance de la herramienta y de las ontologías, y deberán buscarse soluciones en el uso de herramientas automatizadas.\todo{Añadir en la parte de trabajo a futuro}
	
	Las asignaturas optativas quedan, por el momento, fuera del alcance de este trabajo, por no estar definidas en el documento remitido a ANECA. No obstante, tal y como se verá más adelante, es posible crear incorporar al plan de estudios cualquier asignatura optativa, con todas las características de cualquier asignatura, de modo que la materia ``Optatividad'' esté completa. 
	
	Tal y como se especifica en el documento, la inclusión de asignaturas como requisitos de terceras, no especifica una obligatoriedad al uso, y no limita el curso de unas asignaturas antes que otras, sino que es meramente una recomendación del itinerario curricular que debería seguir el alumno. Todas las asignaturas tienen una extensión en créditos ECTS, así como una descripción de sus contenidos.
 
	\item [Actividades formativas.]Son las actividades a realizar por el profesor y el alumnado a lo largo de un curso, diferenciándose unas de otras en el propósito buscado por la acción didáctica. Por tanto, para una misma materia, concurren variadas actividades formativas, ponderándose su distribución a lo largo del curso de dicha materia en función de los objetivos propuestos en el plan de estudios. Las actividades formativas llevan asociadas una extensión en créditos ECTS, que miden el desempeño del alumno en esa actividad en concreto.
 
	\item [Métodos docentes.]Se trata de los conjuntos de formas, procedimientos, técnicas, actividades, etc, de enseñanza y aprendizaje. Para una misma materia, es por tanto compatible el uso de diversos métodos docentes compaginados con las distintas actividades formativas. Algunas actividades formativas y algunos métodos docentes serán mutuamente excluyentes, pero por regla general pueden combinarse entre ellos a criterio del docente. 
 
\end{description}
    
Como ya se comentó al describir las ontologías, su propósito es lograr una descripción lógica del mundo que se está modelando. Por tanto, es posible introducir restricciones del tipo ``toda asignatura debe pertenecer a una materia'', o ``las competencias satisfacen los objetivos establecidos sobre el plan de estudios'', pero no es posible decir, en sintaxis OWL-DL ``la suma de los créditos de una materia debe ser igual a la suma de los créditos de todas las asignaturas que lo componen'', o ``las asignaturas básicas deben sumar 60 créditos''. 	

\section{Explicación de la ontología}

Como ya se ha explicado, las ontologías están compuestas por clases, propiedades e individuos. Las ontologías OWL incluyen además operadores, como unión, intersección o negación, lo que nos permite, además de describir conceptos, definirlos. De este modo, los conceptos más complejos pueden construirse sobre definiciones de otros más simples, lo que facilita la concepción y mantenimiento de estos sistemas. 

Además, el uso de ontologías OWL-DL, nos permite el uso de herramientas automáticas para comprobar que todas las sentencias y definiciones que conforman el lenguaje son consistentes, además de poder razonar qué individuos encajan en qué descripciones, o si la jerarquía establecida es consistente con los individuos presentes y sus descripciones.

Primeramente se mostrarán las clases, pero no se hablará de sus relaciones ni de los individuos que la componen. Posteriormente se tratarán la propiedades de los objetos (sus relaciones), y por último se entrará a detalle con los individuos que componen la ontología, explicando las relaciones, clases y propiedades de ellos.

Antes de continuar con la definición de cada clase y propiedad, vamos a definir la sintaxis utilizada y la semántica asociada. De entre todas las sintaxis definidas en el estándar W3C, se ha optado por utilizar la sintaxis Manchester, por ser de más fácil comprensión que el resto de las sintaxis definidas en el estándar. Para facilitar la comprensión del lector, se incluye en el apéndice una tabla con la semántica de la sintaxis utilizada.

\subsection{Clases de la ontología}
	\lstset{language=protege,basicstyle=\sffamily,columns=flexible,mathescape}
	Las clases se pueden definir en protegé como conjuntos que contienen individuos. Las clases se describen utilizando descripciones formales que definen inequívocamente los requisitos de pertenencia a una clase. 
	
	Las clases se organizan en conjuntos de superclases y subclases, que forman la taxonomía de nuestro universo en observación. Esta taxonomía puede ser obtenida de manera automática por un razonador, que también puede comprobar su consistencia. Se debe tener en cuenta que todas las clases se han definido como disjuntas entre ellas de modo general, salvo las clases \lstinline!Competencia_Especifica! y \lstinline!Competencia_General! cuya disjunción se define de modo particular por ser subclase de \lstinline!Competencia! y ser su unión disjunta (una competencia puede ser general o específica, pero ninguna otra cosa no definida).
  %\lstinputlisting[language=owlms,literate=,firstnumber=671,firstline=671,lastline=672]{t-box.ms}

  
	Se han definido diez clases para la creación de nuestra ontología.	

	\begin{description}
    
	    	%%%%%%%%%%%%%%%%%%%%%%%%%%%%%%%%%%%%%%%%%%%%%%%%%%%%%%%%%%%%%%%%%
    		%% Objetivo_General
	    	%%%%%%%%%%%%%%%%%%%%%%%%%%%%%%%%%%%%%%%%%%%%%%%%%%%%%%%%%%%%%%%%%
		\item[Objetivo\_General.]
		Con la clase \lstinline!Objetivo_General! queremos definir el conjunto de individuos que componen los objetivos generales del título. Esta clase no cuenta con ninguna definición, ya que los objetivos generales vienen definidos por la propia titulación. El único criterio de pertenencia posible a esta clase, es la definición como objetivos de la titulación en el plan de estudios.
		\lstinputlisting[language=owlms,literate=,linerange=Class:\ ects:Objetivo_General-max\ 1\ xsd:string]{t-box.ms}
      
		%%%%%%%%%%%%%%%%%%%%%%%%%%%%%%%%%%%%%%%%%%%%%%%%%%%%%%%%%%%%%%%%%
	    	%% Competencia
    		%%%%%%%%%%%%%%%%%%%%%%%%%%%%%%%%%%%%%%%%%%%%%%%%%%%%%%%%%%%%%%%%%
		\item[Competencia.]
		La clase \lstinline!Competencia! agrupa a todas las competencias que es preciso adquirir para lograr el cumplimiento de los objetivos generales establecidos. Se definen como competencias todos aquellos individuos que representan aptitudes y actitudes, cuya adquisición por parte del alumno conlleva la consecución de los objetivos generales propuestos para la obtención del título.
			
		Formalmente, se define como aquellos individuos que están relacionados con individuos de la clase \lstinline!Objetivo_General! mediante la propiedad que defina la adquisición, al menos una vez y que sólo tengan relación con individuos de la clase\lstinline!Objetivo_General!:
		\lstinputlisting[language=owlms,literate=,firstnumber=595,firstline=595,lastline=608]{t-box.ms}
      
		Dentro de esta clase, podemos definir dos grupos de competencias, según sean competencias específicas (propias de la rama del conocimiento donde se enmarca la titulación) o generales (comunes a todas las ramas del conocimiento y que suelen referirse a aptitudes y actitudes más que a conocimientos o metodologías).

		%%%%%%%%%%%%%%%%%%%%%%%%%%%%%%%%%%%%%%%%%%%%%%%%%%%%%%%%%%%%%%%%%
	    	%% Competencia_General
    		%%%%%%%%%%%%%%%%%%%%%%%%%%%%%%%%%%%%%%%%%%%%%%%%%%%%%%%%%%%%%%%%%
		\item[Competencia\_General.] Son competencias generales todos aquellos individuos que son competencias y que además adquiere el alumno al cursar una materia definida. Dicho de otro modo, el hecho de superar una determinada materia, demuestra que el alumno ha adquirido ciertas competencias generales.
		%\lstinputlisting[language=owlms,literate=,linerange=Class:\ ects:Competencia_General-Competencia_Especifica]{t-box.ms}	

		%%%%%%%%%%%%%%%%%%%%%%%%%%%%%%%%%%%%%%%%%%%%%%%%%%%%%%%%%%%%%%%%%
	    	%% Competencia_Especifica
    		%%%%%%%%%%%%%%%%%%%%%%%%%%%%%%%%%%%%%%%%%%%%%%%%%%%%%%%%%%%%%%%%%
		\item[Competencia\_Especifica.] La clase \lstinline!Competencia_Especifica! agrupa aquellas competencias que es preciso adquirir para el cumplimiento de los objetivos generales del título, y que son específicos de la rama del conocimiento propia de la titulación.	

		Se consideran competencias específicas todos aquellos individuos que son competencias y que además se adquieren durante el curso de una materia. Visto de una manera más formal:
		\lstinputlisting[language=owlms,literate=,linerange=Class:\ ects:Competencia_Especifica-Competencia_General]{t-box.ms}	
	
		Como hemos visto, competencias generales y específicas son disjuntas entre sí, pero para la especificación de una \lstinline!Compentencia! dentro de \lstinline!Competencia_General! o \lstinline!Competencia_Especifica! es suficiente con que la relación de la \lstinline!Competencia! con la \lstinline!Materia! tenga como dominio uno u otro conjunto de competencias, sin mayor diferenciación.
  
  		Visto de esta forma, se podría haber optado por eliminar las subclases \lstinline!Competencia_General! y \lstinline!Competencia_Especifica! y que hubiesen quedado diferenciadas únicamente por una etiqueta (con una Data Property diferente, por ejemplo), pero por cuestiones de diseño, y para facilitar la comprensión del modelo se ha optado por separar en dos clases las competencias específicas y generales. 
  
  		¿Cómo distinguirá protegé entre competencias específicas y generales? Los individuos pertenecientes a cada una de las clases, se relacionarán con las materias mediante dos relaciones distintas, creadas \textit{ad hoc} para cada una de las clases. De este modo logramos que el razonador, una vez haya identificado un individuo como perteneciente a la clase competencia, pueda clasificarlo como \lstinline!Competencia_General! o \lstinline!Competencia_Especifica! en función de la propiedad que lo relacione con las materias, ganando en potencia de cálculo, aunque para ello la ontología deba de ser algo más compleja.
      
      
		%%%%%%%%%%%%%%%%%%%%%%%%%%%%%%%%%%%%%%%%%%%%%%%%%%%%%%%%%%%%%%%%%
	    	%% Materia
    		%%%%%%%%%%%%%%%%%%%%%%%%%%%%%%%%%%%%%%%%%%%%%%%%%%%%%%%%%%%%%%%%%
		\item[Materia.]La clase \lstinline!Materia! aglutina todos los individuos que representan las diferentes materias de que se compone la titulación. Coloquialmente podemos entender las materias como aquellos individuos que permiten al alumno adquirir competencias, de modo que puedan cumplir con las objetivos generales del título. Formalmente hablando, son materias todos aquellos individuos relacionados con algún individuo de la clase Competencia, y al menos una vez con alguna \lstinline!Competencia_General!. Posteriormente se definirá la propiedad que une \lstinline!Materia! con \lstinline!Compentencia! y con \lstinline!Competencia_General!. 
		\lstinputlisting[language=owlms,literate=,linerange=Class:\ ects:Materia-only\ ects:Metodo_Docente]{t-box.ms}	
      
      
      
      
		%%%%%%%%%%%%%%%%%%%%%%%%%%%%%%%%%%%%%%%%%%%%%%%%%%%%%%%%%%%%%%%%%
    		%% Asignatura
	    	%%%%%%%%%%%%%%%%%%%%%%%%%%%%%%%%%%%%%%%%%%%%%%%%%%%%%%%%%%%%%%%%%
		\item[Asignatura.]En la clase \lstinline!Asignatura! se agrupan todos los individuos que representan las diferentes asignaturas de que se compone cada materia. 
		
		Las definiciones de asignatura y de materia son complementarias: Podemos definir una materia como el conjunto de asignaturas de un mismo ámbito, o bien podemos definir una asignatura como parte integrante de una materia, con quien comparte la naturaleza de los conocimientos contenidos. Dado el carácter más general de la materia, se ha optado en la ontología por que sean las materias a partir de las cuales se definan las asignaturas, y no al contrario. 
		
      	Además, esa decisión simplifica el modelo, ya que como veremos más adelante, también actividades formativas y métodos docentes dependerán para su definición en el modelo de la clase \lstinline!Materia!. Formalmente descrito, podemos definir una asignatura como todo aquel individuo relacionado mediante la propiedad que indica la inclusión con algún individuo de la clase \lstinline!Materia! al menos una vez y sólo con indivudos que sean ``Materias'', de modo que la asignatura forme parte de dicha Materia.
      	\lstinputlisting[language=owlms,literate=,linerange=Class:\ ects:Asignatura-only\ ects:Caracter]{t-box.ms}	
      
      
		%%%%%%%%%%%%%%%%%%%%%%%%%%%%%%%%%%%%%%%%%%%%%%%%%%%%%%%%%%%%%%%%%
		%% Actividad_Formativa
		%%%%%%%%%%%%%%%%%%%%%%%%%%%%%%%%%%%%%%%%%%%%%%%%%%%%%%%%%%%%%%%%%
		\item[Actividad\_Formativa.]La clase \lstinline!Actividad_Formativa! define todos aquellos individuos que tienen su correspondencia en el mundo real con las distintas actividades formativas que pueden desarrollarse para el aprendizaje de la asignatura. Una actividad formativa es la actividad a realizar por el profesor y el alumnado a lo largo de un curso, diferenciándose unas de otras en el propósito buscado por la acción didáctica. Por tanto, para una misma materia, concurren variadas actividades formativas, ponderándose su distribución a lo largo del curso de dicha materia en función de los objetivos propuestos en el plan de estudios.
		
		Por tanto, de cara a nuestra ontología, definiremos la clase actividad formativa como el conjunto de individuos que representa actividades didácticas, que se utilizan  para impartir una o varias materias. En términos formales, serían todos los individuos que están relacionados al menos una vez con al menos un individuo de clase \lstinline!Materia!. 
		
		Como ya se ha comentado anteriormente, las actividades formativas llevan asociados un número de créditos. Por tanto, para poder almacenar esa información se ha optado por crear una clase \lstinline!Actividad_Formativa! y luego crear un individuo por cada par \lstinline!Actividad_Formativa!-\lstinline!Materia!. Los créditos ECTS asociados a cada actividad formativa se almacenarán como un \textit{data property} de cada individuo. 
		
		La otra opción, tener únicamente un individuo por cada \lstinline!Actividad_Formativa! que esté relacionada con las materias, no permite guardar la información de los créditos ECTS, ya que en el caso de una misma actividad formativa utilizada para la docencia de dos asignaturas diferentes, nos sería imposible discernir si la duración de dicha actividad formativa se refiere a la utilizada para la impartición de una u otra asignatura.
		\lstinputlisting[language=owlms,literate=,firstnumber=720,firstline=720,lastline=731]{t-box.ms}	

		%%%%%%%%%%%%%%%%%%%%%%%%%%%%%%%%%%%%%%%%%%%%%%%%%%%%%%%%%%%%%%%%%
		%% Metodo_Docente
		%%%%%%%%%%%%%%%%%%%%%%%%%%%%%%%%%%%%%%%%%%%%%%%%%%%%%%%%%%%%%%%%%
		\item[Metodo\_Docente.]En la ontología definimos la clase \lstinline!Metodo_Docente! como el conjunto de procedimientos y técnicas de enseñanza y aprendizaje utilizados para impartir una materia. Más formalmente, podríamos definir un método docente como aquellos individuos que están relacionados al menos una vez con al menos un individuo de la clase materia. 
		\lstinputlisting[language=owlms,literate=,linerange=Class:\ ects:Metodo_Docente-max\ 1\ xsd:string]{t-box.ms}	


		%%%%%%%%%%%%%%%%%%%%%%%%%%%%%%%%%%%%%%%%%%%%%%%%%%%%%%%%%%%%%%%%%
		%% Metodo_Evaluacion
		%%%%%%%%%%%%%%%%%%%%%%%%%%%%%%%%%%%%%%%%%%%%%%%%%%%%%%%%%%%%%%%%%
		\item[Metodo\_Evaluacion.]Al igual que ocurre con los métodos docentes, existen múltiples variedades de métodos de evaluación que permiten comprobar si el alumno ha adquirido las competencias necesarias para que se le otorguen los créditos de las materias cursadas. Con el fin de adecuar los métodos de evaluación ópitmos a cada una de las materias del plan, se ha optado por recoger en una clase individual las distintas modalidades de evaluación que se aplicarán para conocer el aprovechamiento de una materia.
		\lstinputlisting[language=owlms,literate=,linerange=Class:\ ects:Metodo_Evaluacion-max\ 1\ xsd:string]{t-box.ms}	


		%%%%%%%%%%%%%%%%%%%%%%%%%%%%%%%%%%%%%%%%%%%%%%%%%%%%%%%%%%%%%%%%%
		%% Ubicacion_Temporal
		%%%%%%%%%%%%%%%%%%%%%%%%%%%%%%%%%%%%%%%%%%%%%%%%%%%%%%%%%%%%%%%%%
		\item[Ubicacion\_Temporal.]Cada una de las materias y de las asignaturas que las componen, tienen asignadas una ubicación temporal dentro del itinerario formativo. Así, las materias y asignaturas se reparten a lo largo de los semestres, por lo que resulta útil conocer que carga de trabajo (es decir, el número de créditos ECTS) compone cada trimestre.

		Además de las asignaturas, las materias también tienen asigando una ubicación temporal, que dependerá de las ubicaciones de las asignaturas que la compongan. Es decir, si una materia está compuesta de dos asignaturas, y una asignatura está asignada al primer semestre y la otra asignatura lo está al segundo semestre, la materia queda ubicada en el primer y segundo semestre.
		\lstinputlisting[language=owlms,literate=,firstnumber=674,firstline=674,lastline=688]{t-box.ms}	
    	
    	
		%%%%%%%%%%%%%%%%%%%%%%%%%%%%%%%%%%%%%%%%%%%%%%%%%%%%%%%%%%%%%%%%%
		%% Caracter
		%%%%%%%%%%%%%%%%%%%%%%%%%%%%%%%%%%%%%%%%%%%%%%%%%%%%%%%%%%%%%%%%%
		\item[Caracter.]	El carácter de una asignatura viene definido por el organismo que impone su estudio para una titulación. Pero al contrario que en el caso de las ubicaciones, no podemos decir que si una materia está compuesta de asignaturas básicas, ella sea de carácter básico, pues en el documento remitido a ANECA, hemos encontrado casos de materias (estadística, pej.) en los que una materia compuesta de asignaturas básicas y obligatorias es de carácter básico, y no mixto como a priori podría parecer. 
		\lstinputlisting[language=owlms,literate=,firstnumber=626,firstline=626,lastline=640]{t-box.ms}	
    	
    	
	\end{description}

	\subsection{Propiedades de la ontología}
	Las propiedades son relaciones binarias entre individuos, es decir, una propiedad une dos individuos entre sí. Las propiedades pueden tener inversas, pueden ser funcionales, transitivas, simétricas\ldots  Estas relaciones se pueden dar tanto entre individuos de la misma clase, como entre individuos de distintas clases.
	Un razonador automático puede computar si una relación entre dos individuos es consistente con el resto de la ontología. 
	Es de destacar que las propiedades únicamente se pueden establecer entre dos individuos. No existen propiedades con cardinalidad tres, lo que implicará que en el caso de que que sea preciso establecer una relación a tres, sería preciso modelarlo como la relación binaria entre el producto escalar de dos de ellas sobre la tercera (tal y como se comentó al mencionar la clase \lstinline!Actividad_Formativa! y la forma de almacenar la cantidad de créditos otorgados).
  
	\begin{description}

	   	%%%%%%%%%%%%%%%%%%%%%%%%%%%%%%%%%%%%%%%%%%%%%%%%%%%%%%%%%%%%%%%%%
   		%% OG\_seCumpleMedianteLaAdquisicionDe\_CO y CO\_seAdquiereParaCumplir\_OG
	   	%%%%%%%%%%%%%%%%%%%%%%%%%%%%%%%%%%%%%%%%%%%%%%%%%%%%%%%%%%%%%%%%%
    		\item [OG\_seCumpleMedianteLaAdquisicionDe\_CO y CO\_seAdquiereParaCumplir\_OG.] Como ya se comentó anteriormente, el objetivo de las titulaciones de grado es lograr que el alumno adquiera unos niveles de destreza mínimos en el manejo de ciertas competencias definidas en el título, de modo que esté capacitado para enfrentarse al mercado laboral.
    
	    Estas competencias se van adquiriendo conforme el alumno avanza por el itinerario formativo, hasta alcanzar los niveles mínimos exigidos en la titulación al finalizar sus estudios. Es por tanto la adquisición de esas competencias lo que permite al alumno cumplir con las objetivos generales establecidos en el título de grado. Dicho de otro modo, la titulación permite el cumplimiento de los objetivos marcados, en la medida que permite al alumno ir adquiriendo las competencias precisas para ello.
		\lstinputlisting[language=owlms,literate=,linerange=ObjectProperty:\ ects:OG_seCumpleMedianteLaAdquisicionDe_CO-ects:CO_seAdquiereParaCumplir_OG]{t-box.ms}	
		\lstinputlisting[language=owlms,literate=,linerange=ObjectProperty:\ ects:CO_seAdquiereParaCumplir_OG-ects:OG_seCumpleMedianteLaAdquisicionDe_CO]{t-box.ms}	
    
	   	%%%%%%%%%%%%%%%%%%%%%%%%%%%%%%%%%%%%%%%%%%%%%%%%%%%%%%%%%%%%%%%%%
   		%% CO_esOtorgadaPor_MA y MA_otorgaCompetencias_CO
	   	%%%%%%%%%%%%%%%%%%%%%%%%%%%%%%%%%%%%%%%%%%%%%%%%%%%%%%%%%%%%%%%%%    
	    \item [CO\_esOtorgadaPor\_MA y MA\_otorgaCompetencias\_CO.] Los alumnos adquieren las competencias conforme van superando con éxito las materias del plan de estudio, lo que les permite, una vez superadas todas las materias, cumplir con los objetivos generales del título.
    		\lstinputlisting[language=owlms,literate=,linerange=ObjectProperty:\ ects:CG_esOtorgadaPor_MA-ects:MA_otorgaCompetenciasGenerales_CG]{t-box.ms}	
		\lstinputlisting[language=owlms,literate=,linerange=ObjectProperty:\ ects:MA_otorgaCompetencias_CO-ects:CO_esOtorgadaPor_MA]{t-box.ms}	
    
 
	   	%%%%%%%%%%%%%%%%%%%%%%%%%%%%%%%%%%%%%%%%%%%%%%%%%%%%%%%%%%%%%%%%%
   		%% CG\_esOtorgadaPor\_MA y MA\_otorgaCompetenciasGenerales\_CG
	   	%%%%%%%%%%%%%%%%%%%%%%%%%%%%%%%%%%%%%%%%%%%%%%%%%%%%%%%%%%%%%%%%%
    		\item [CG\_esOtorgadaPor\_MA y MA\_otorgaCompetenciasGenerales\_CG.] Es obligatorio que el alumno, al finalizar la titulación haya adquirido ciertas compentencias que resultan básicas para un profesional en el campo de la titulación. Estas competencias obligatorias reciben el nombre de competencias generales, y son las líneas maestras que definirán el progreso del alumno en la titulación.
	    \lstinputlisting[language=owlms,literate=,linerange=ObjectProperty:\ ects:CG_esOtorgadaPor_MA-ects:MA_otorgaCompetenciasGenerales_CG]{t-box.ms}	
	    	\lstinputlisting[language=owlms,literate=,linerange=ObjectProperty:\ ects:MA_otorgaCompetenciasGenerales_CG-ects:CG_esOtorgadaPor_MA]{t-box.ms}	
    
	   	%%%%%%%%%%%%%%%%%%%%%%%%%%%%%%%%%%%%%%%%%%%%%%%%%%%%%%%%%%%%%%%%%
   		%% CE\_esOtorgadaPor\_MA y MA\_otorgaCompetenciasEspecíficas\_CE
	   	%%%%%%%%%%%%%%%%%%%%%%%%%%%%%%%%%%%%%%%%%%%%%%%%%%%%%%%%%%%%%%%%%
    		\item [CE\_esOtorgadaPor\_MA y MA\_otorgaCompetenciasEspecíficas\_CE.] Al igual que ocurre con las competencias generales, también es preciso que un alumno adquiera las competencias específicas propias de la titulación para cumplir los objetivos generales de esa titulación. Y al igual que ocurre con las competencias generales, las específicas únicamente se obtienen cuando el alumno supera con éxito las materias de la titulación.
    		\lstinputlisting[language=owlms,literate=,linerange=ObjectProperty:\ ects:CE_esOtorgadaPor_MA-ects:MA_otorgaCompetenciasEspecificas_CE]{t-box.ms}
        \lstinputlisting[language=owlms,literate=,linerange=ObjectProperty:\ ects:MA_otorgaCompetenciasEspecificas_CE-ects:CE_esOtorgadaPor_MA]{t-box.ms}

	    %%%%%%%%%%%%%%%%%%%%%%%%%%%%%%%%%%%%%%%%%%%%%%%%%%%%%%%%%%%%%%%%%
   		%% MA\_constaDe\_AS y AS\_formaParteDe\_MA
	   	%%%%%%%%%%%%%%%%%%%%%%%%%%%%%%%%%%%%%%%%%%%%%%%%%%%%%%%%%%%%%%%%%
    		\item [MA\_constaDe\_AS y AS\_formaParteDe\_MA.] Una materia está compuesta por varias asignaturas, que quedan de ese modo agrupadas bajo esa materia. Se podría decir de otro modo: que las asignaturas quedan agrupadas en diversas materias. Es una relación inversamente funcional, es decir, su función inversa es funcional. Dicho coloquialmente, significa que una materia puede constar de una o varias asignaturas, pero una asignatura sólo puede pertenecer a una materia.
        \lstinputlisting[language=owlms,literate=,linerange=ObjectProperty:\ ects:MA_constaDe_AS-ects:AS_formaParteDe_MA]{t-box.ms}
        \lstinputlisting[language=owlms,literate=,linerange=ObjectProperty:\ ects:AS_formaParteDe_MA-ects:MA_constaDe_AS]{t-box.ms}
    
	    %%%%%%%%%%%%%%%%%%%%%%%%%%%%%%%%%%%%%%%%%%%%%%%%%%%%%%%%%%%%%%%%%
   		%% MA\_seImparteMediante\_ME y ME\_utilizadoParaImpartir\_MA
	   	%%%%%%%%%%%%%%%%%%%%%%%%%%%%%%%%%%%%%%%%%%%%%%%%%%%%%%%%%%%%%%%%%    
    		\item [MA\_seImparteMediante\_ME y ME\_utilizadoParaImpartir\_MA.] Las materias se imparten según dictan ciertos métodos docentes considerados adecuados por los docentes de la titulación. Estos métodos docentes pueden ser complementarios entre sí, y en ningún caso son excluyentes entre ellos. De este modo se logra que la adquisición de competencias por parte del alumno sea óptima.
        \lstinputlisting[language=owlms,literate=,linerange=ObjectProperty:\ ects:MA_seImparteMediante_ME-ects:ME_utilizadoParaImpartir_MA]{t-box.ms}
        \lstinputlisting[language=owlms,literate=,linerange=ObjectProperty:\ ects:ME_utilizadoParaImpartir_MA-ects:MA_seImparteMediante_ME]{t-box.ms}
    
	    %%%%%%%%%%%%%%%%%%%%%%%%%%%%%%%%%%%%%%%%%%%%%%%%%%%%%%%%%%%%%%%%%
   		%% MA\_seImparteSegún\_AF y AF\_utilizadaParaImpartir\_MA
	   	%%%%%%%%%%%%%%%%%%%%%%%%%%%%%%%%%%%%%%%%%%%%%%%%%%%%%%%%%%%%%%%%% 
   		\item [MA\_seImparteSegún\_AF y AF\_utilizadaParaImpartir\_MA.] Las materias se imparten realizando ciertas actividades formativas definidas por los docentes de la materia. Los créditos de docencia de esa materia, deben estar por tanto distribuidos entre las distintas actividades formativas. Al igual que ocurre con los métodos docentes, las diferentes actividades formativas no son excluyentes entre sí, sino complementarios. Además se ha considerado importante, de cara al diseño del título, conocer cuántos créditos se dedican a cada actividad formativa. Más adelante, veremos cómo se almacena esa información.
        \lstinputlisting[language=owlms,literate=,linerange=ObjectProperty:\ ects:MA_seImparteSegun_AF-ects:AF_utilizadaParaImpartir_MA]{t-box.ms}
        \lstinputlisting[language=owlms,literate=,linerange=ObjectProperty:\ ects:AF_utilizadaParaImpartir_MA-ects:MA_seImparteSegun_AF]{t-box.ms}

	    %%%%%%%%%%%%%%%%%%%%%%%%%%%%%%%%%%%%%%%%%%%%%%%%%%%%%%%%%%%%%%%%%
   		%% MA_tiene_CA y CA_agrupa_MA
	   	%%%%%%%%%%%%%%%%%%%%%%%%%%%%%%%%%%%%%%%%%%%%%%%%%%%%%%%%%%%%%%%%% 
		\item[MA\_tiene\_CA y CA\_agrupa\_MA.] Las materias pueden ser básicas, obligatorias, optativas o mixtas. Este carácter de las materias dependerá en gran medida (aunque no de forma única) del carácter de las asignaturas que lo compongan.
		\lstinputlisting[language=owlms,literate=,linerange=ObjectProperty:\ ects:MA_tiene_CA-ects:CA_agrupa_MA]{t-box.ms}
        \lstinputlisting[language=owlms,literate=,linerange=ObjectProperty:\ ects:CA_agrupa_MA-ects:MA_tiene_CA]{t-box.ms}

	    %%%%%%%%%%%%%%%%%%%%%%%%%%%%%%%%%%%%%%%%%%%%%%%%%%%%%%%%%%%%%%%%%
   		%% MA_ubicada_UT y UT_contiene_MA
	   	%%%%%%%%%%%%%%%%%%%%%%%%%%%%%%%%%%%%%%%%%%%%%%%%%%%%%%%%%%%%%%%%% 
		\item[MA\_ubicada\_UT y UT\_contiene\_MA.] El desarrollo de la docencia se estructura en trimestres. Estas ubicaciones temporales son útiles a la hora de estructurar el desarrollo de la docencia y adecuarlo a la duración de la misma.
		\lstinputlisting[language=owlms,literate=,linerange=ObjectProperty:\ ects:MA_ubicada_UT-ects:UT_contiene_MA]{t-box.ms}
        \lstinputlisting[language=owlms,literate=,linerange=ObjectProperty:\ ects:UT_contiene_MA-ects:MA_ubicada_UT]{t-box.ms}
    				
	    %%%%%%%%%%%%%%%%%%%%%%%%%%%%%%%%%%%%%%%%%%%%%%%%%%%%%%%%%%%%%%%%%
   		%% MA_utiliza_MV y MV_esUtilizadoEn_MA
	   	%%%%%%%%%%%%%%%%%%%%%%%%%%%%%%%%%%%%%%%%%%%%%%%%%%%%%%%%%%%%%%%%% 
		\item[MA\_utiliza\_MV y MV\_esUtilizadoEn\_MA.] Cada una de las materias tiene pruebas de evaluación idóneas, que serán las utilizadas para comprobar el correcto aprovechamiento del curso de esas materias por parte del alumno, es decir, la adquisición de las competencias indicadas en el grado adecuado.
		\lstinputlisting[language=owlms,literate=,linerange=ObjectProperty:\ ects:MA_utiliza_MV-ects:MV_esUtilizadoEn_MA]{t-box.ms}
        \lstinputlisting[language=owlms,literate=,linerange=ObjectProperty:\ ects:MV_esUtilizadoEn_MA-ects:MA_utiliza_MV]{t-box.ms}

	    %%%%%%%%%%%%%%%%%%%%%%%%%%%%%%%%%%%%%%%%%%%%%%%%%%%%%%%%%%%%%%%%%
   		%% AS\_esRequisitoPara\_AS y AS\_tieneComoRequisito\_AS
	   	%%%%%%%%%%%%%%%%%%%%%%%%%%%%%%%%%%%%%%%%%%%%%%%%%%%%%%%%%%%%%%%%% 
    		\item [AS\_esRequisitoPara\_AS y AS\_tieneComoRequisito\_AS.] Como ya se explicó antes, el hecho de que una asignatura tenga como requisito en la guía de la titulación el haber cursado una asignatura anteriormente, no implica que obligatoriamente se haya de cursar esa asignatura con anterioridad. Es decir, se trata de una mera recomendación de cara al itinerio a seguir en la titulación, y en ningún caso de obligado cumplimiento. Se ha tenido en cuenta a la hora de elaborar la ontología, ya que el fín último de esta es proporcionar una herramienta para un diseño de la titulación correcto, lo que incluye el trayecto curricular del alumno, y no para el control del desarrollo del alumno. Por tanto, dado que lo que estamos haciendo es ayudar a diseñar el título, vamos a incluir los requisitos para el curso de asignaturas, como si de cumplimiento obligatorio se tratase.
    		
    		Es una relación transitiva, es decir, que si la asignatura ``A'' es requisito de la asignatura ``B'' y a su vez ``B'' es requisito de la asignatura ``C'', entonces la asignatura ``A'' es un requisito de la asignatura ``C''.
	    \lstinputlisting[language=owlms,literate=,linerange=ObjectProperty:\ ects:AS_esRequisitoPara_AS-ects:AS_tieneComoRequisito_AS]{t-box.ms}
        \lstinputlisting[language=owlms,literate=,linerange=ObjectProperty:\ ects:AS_tieneComoRequisito_AS-ects:AS_esRequisitoPara_AS]{t-box.ms}
    
	    %%%%%%%%%%%%%%%%%%%%%%%%%%%%%%%%%%%%%%%%%%%%%%%%%%%%%%%%%%%%%%%%%
   		%% AS_tiene_CA y CA_agrupa_AS
	   	%%%%%%%%%%%%%%%%%%%%%%%%%%%%%%%%%%%%%%%%%%%%%%%%%%%%%%%%%%%%%%%%% 
    		\item [AS\_tiene\_CA y CA\_agrupa\_AS.] Las asignaturas también tienen diferentes carácteres, lo que a su vez permite dotar de carácter a las materias que las engloban.
	    \lstinputlisting[language=owlms,literate=,linerange=ObjectProperty:\ ects:AS_tiene_CA-ects:CA_agrupa_AS]{t-box.ms}
	    \lstinputlisting[language=owlms,literate=,linerange=ObjectProperty:\ ects:CA_agrupa_AS-ects:AS_tiene_CA]{t-box.ms}
    
	    %%%%%%%%%%%%%%%%%%%%%%%%%%%%%%%%%%%%%%%%%%%%%%%%%%%%%%%%%%%%%%%%%
   		%% AS_ubicada_UT y UT_contiene_AS
	   	%%%%%%%%%%%%%%%%%%%%%%%%%%%%%%%%%%%%%%%%%%%%%%%%%%%%%%%%%%%%%%%%% 
    		\item [AS\_ubicada\_UT y UT\_contiene\_AS.] Las asignaturas se programan para que se desarrollen en trimestres, de modo que el alumno pueda cursar las asignaturas de un modo ordenado, lo que hará que el itinerario formativo sea estructurado. Una adecuada estructuración del itinerario formativo sólo es posible mediante el uso de un diseño adecuado.
	    \lstinputlisting[language=owlms,literate=,linerange=ObjectProperty:\ ects:AS_ubicada_UT-ects:UT_contiene_AS]{t-box.ms}
		\lstinputlisting[language=owlms,literate=,linerange=ObjectProperty:\ ects:UT_contiene_AS-ects:AS_ubicada_UT]{t-box.ms}
    
    
	    %%%%%%%%%%%%%%%%%%%%%%%%%%%%%%%%%%%%%%%%%%%%%%%%%%%%%%%%%%%%%%%%%
   		%% CA_agrupa_ASMA y UT_contiene_ASMA
	   	%%%%%%%%%%%%%%%%%%%%%%%%%%%%%%%%%%%%%%%%%%%%%%%%%%%%%%%%%%%%%%%%% 
   	 	\item [CA\_agrupa\_ASMA y UT\_contiene\_ASMA.] Es interesante, como herramienta para poder realizar un diseño coherente del itinerario formativo, conocer la estructuración temporal y de carácter de las diferentes asignaturas y materias de que consta la titulación, pues de este modo se pueden distribuir adecuadamente las asignaturas a lo largo del itinerario formativo, así como conocer la composición de asignaturas y materias
	    \lstinputlisting[language=owlms,literate=,linerange=ObjectProperty:\ ects:CA_agrupa_ASMA-or\ ects:Materia]{t-box.ms}
        \lstinputlisting[language=owlms,literate=,linerange=ObjectProperty:\ ects:UT_contiene_ASMA-or\ ects:Materia]{t-box.ms}
    
	\end{description}
  
	\subsection{Propiedades sobre tipos de datos}
	Las propiedaddes sobre tipos de datos son relaciones entre individuos y tipos de datos, de modo que se puede asociar a los individuos implicados en la relación ciertas características de tipos definidos, concretas y especificas.
  
	\begin{description}
    		%%%%%%%%%%%%%%%%%%%%%%%%%%%%%%%%%%%%%%%%%%%%%%%%%%%%%%%%%%%%%%%%%
	   	%% OG_Descripcion
   		%%%%%%%%%%%%%%%%%%%%%%%%%%%%%%%%%%%%%%%%%%%%%%%%%%%%%%%%%%%%%%%%% 
		\item [OG\_Descripcion.] Se trata de una relación entre un individuo de la clase \lstinline!Objetivo\_General! y un dato de tipo literal. Esta relación funcional (un objetivo sólo puede tener una descripción) es útil para guardar la descripción del objetivo descrito, y poder identificarlo por algo menos formal y más útil que la numeración del objetivo.
   		\lstinputlisting[language=owlms,literate=,linerange=DataProperty:\ ects:OG_Descripcion-xsd:string]{t-box.ms}

	    %%%%%%%%%%%%%%%%%%%%%%%%%%%%%%%%%%%%%%%%%%%%%%%%%%%%%%%%%%%%%%%%%
   		%% CO_Descripcion
	   	%%%%%%%%%%%%%%%%%%%%%%%%%%%%%%%%%%%%%%%%%%%%%%%%%%%%%%%%%%%%%%%%%    
	   \item [CO\_Descripcion.] Esta propiedad relaciona una competencia con un literal. También es una propiedad funcional, y al igual que el resto de propiedades dedicadas a almacenar información acerca de descripciones de individuos, es útil para poder identificar los mismos por algo más que el código descriptivo.
   		\lstinputlisting[language=owlms,literate=,linerange=DataProperty:\ ects:CO_Descripcion-xsd:string]{t-box.ms}
  
	    %%%%%%%%%%%%%%%%%%%%%%%%%%%%%%%%%%%%%%%%%%%%%%%%%%%%%%%%%%%%%%%%%
   		%% MA_Creditos
	   	%%%%%%%%%%%%%%%%%%%%%%%%%%%%%%%%%%%%%%%%%%%%%%%%%%%%%%%%%%%%%%%%%   
	   	\item [MA\_Creditos.] Se utliza para saber qué cantidad de créditos otorga al alumno el curso de una materia, o lo que es lo mismo, el coste en créditos ECTS de cursar una materia. Como todas la propiedades que sólo pueden aparecer asociadas a un mismo individuo una vez, se trata de una propiedad funcional.
   		\lstinputlisting[language=owlms,literate=,linerange=DataProperty:\ ects:MA_Creditos-tipoCredito]{t-box.ms} 
   
	    %%%%%%%%%%%%%%%%%%%%%%%%%%%%%%%%%%%%%%%%%%%%%%%%%%%%%%%%%%%%%%%%%
   		%% MA_Coordinacion
	   	%%%%%%%%%%%%%%%%%%%%%%%%%%%%%%%%%%%%%%%%%%%%%%%%%%%%%%%%%%%%%%%%% 
	   \item [MA\_Coordinacion.] Esta ``type property'' se utiliza como un campo libre donde se define de manera informal los agentes que coordinarán la materia, y cuando se defina, las herramentas que utilizarán en su cometido.
		\lstinputlisting[language=owlms,literate=,linerange=DataProperty:\ ects:MA_Coordinacion-xsd:string]{t-box.ms} 

	    %%%%%%%%%%%%%%%%%%%%%%%%%%%%%%%%%%%%%%%%%%%%%%%%%%%%%%%%%%%%%%%%%
   		%% MA_Resultados
	   	%%%%%%%%%%%%%%%%%%%%%%%%%%%%%%%%%%%%%%%%%%%%%%%%%%%%%%%%%%%%%%%%% 
		\item [MA\_Resultados.] Esta relación sólo enumerará a título informativo, los resultados del aprendizaje obtenidos, o dicho de otro modo, el nivel de adquisición de competencias una vez finalizado el curso de la materia objeto de estudio.  
   		\lstinputlisting[language=owlms,literate=,linerange=DataProperty:\ ects:MA_Resultados-xsd:string]{t-box.ms}
			
	    %%%%%%%%%%%%%%%%%%%%%%%%%%%%%%%%%%%%%%%%%%%%%%%%%%%%%%%%%%%%%%%%%
   		%% AS_Creditos
	   	%%%%%%%%%%%%%%%%%%%%%%%%%%%%%%%%%%%%%%%%%%%%%%%%%%%%%%%%%%%%%%%%%   
		\item [AS\_Creditos.] Cada asignatura tiene asociado un coste en créditos ECTS, que se otorgarán al alumno una vez superada.
   		\lstinputlisting[language=owlms,literate=,linerange=DataProperty:\ ects:AS_Creditos-ects:tipoCredito]{t-box.ms}
			
	    %%%%%%%%%%%%%%%%%%%%%%%%%%%%%%%%%%%%%%%%%%%%%%%%%%%%%%%%%%%%%%%%%
   		%% AS_Contenidos
	   	%%%%%%%%%%%%%%%%%%%%%%%%%%%%%%%%%%%%%%%%%%%%%%%%%%%%%%%%%%%%%%%%%   
		\item [AS\_Contenidos.] La relación \lstinline!AS\_Contenidos!, permite relacionar un individuo de la clase asignatura con un dato literal, de modo que nos permite almacenar una descripción algo más extensa, que ayude en su identificación.  
		\lstinputlisting[language=owlms,literate=,linerange=DataProperty:\ ects:AS_Contenidos-xsd:string]{t-box.ms}   



	    %%%%%%%%%%%%%%%%%%%%%%%%%%%%%%%%%%%%%%%%%%%%%%%%%%%%%%%%%%%%%%%%%
   		%% ME_Descripcion
	   	%%%%%%%%%%%%%%%%%%%%%%%%%%%%%%%%%%%%%%%%%%%%%%%%%%%%%%%%%%%%%%%%%
		\item [ME\_Descripcion.] Se trata de una relación entre un individuo de la clase \lstinline!Metodo\_Docente! y un dato de tipo literal. Esta relación funcional (un método docente sólo puede tener una descripción) es útil para guardar la descripción del método descrito.
   		\lstinputlisting[language=owlms,literate=,linerange=DataProperty:\ ects:ME_Descripcion-xsd:string]{t-box.ms}   
			
	    %%%%%%%%%%%%%%%%%%%%%%%%%%%%%%%%%%%%%%%%%%%%%%%%%%%%%%%%%%%%%%%%%
   		%% AF_Creditos
	   	%%%%%%%%%%%%%%%%%%%%%%%%%%%%%%%%%%%%%%%%%%%%%%%%%%%%%%%%%%%%%%%%%
		\item [AF\_Creditos.] Es preciso, de cara a la construcción y mantenimiento del título, conocer qué tiempo ha de dedicar el alumno a cada actividad formativa, con el fin de asegurar la correcta adquisición de las competencias en el tiempo establecido.			
		\lstinputlisting[language=owlms,literate=,linerange=DataProperty:\ ects:AF_Creditos-ects:tipoCredito]{t-box.ms}   

	    %%%%%%%%%%%%%%%%%%%%%%%%%%%%%%%%%%%%%%%%%%%%%%%%%%%%%%%%%%%%%%%%%
   		%% MV_Descripcion
	   	%%%%%%%%%%%%%%%%%%%%%%%%%%%%%%%%%%%%%%%%%%%%%%%%%%%%%%%%%%%%%%%%%    
		\item [MV\_Descripcion.] Recoge una pequeña descripción informal del método de evaluación con el que está relacionado, de modo que sea fácil comprobar las pruebas y técnicas de evaluación que el docente utilizará a la hora de verificar el nivel de adquisición de competencias por parte de los alumnos.
   		\lstinputlisting[language=owlms,literate=,linerange=DataProperty:\ ects:MV_Descripcion-xsd:string]{t-box.ms} 
	\end{description}
  
\subsection{Definiciones de tipos de datos}
Se ha definido un nuevo tipo de datos, más por comodidad a la hora de definir las clases y propiedades, que por que sea realmente necesario a la hora del desarrollo de la ontología. En el caso de que definiésemos un tipo de datos cuya primitiva no fuese alguno de los tipos de datos definidos en el estándar de OWL2, el lenguaje de la ontología pasaría a ser OWL-Full, con lo que no podríamos realizar razonamiento alguno sobre dicha ontología, perdiendo una cualidad básica de las que buscamos con el desarrollo del presente trabajo.

El tipo de datos \lstinline!tipoCredito! nos es útil para poder almacenar los créditos de materias y asignaturas, y si bien puede ser prescindible, me ha parecido útil restringir el rango de los datos únicamente al conjunto de los decimales positivos como una forma de evitar que el usuario de la ontología pueda introducir datos erróneos.

El resto de tipos de datos utilizados, son los primitivos del lenguaje xml, y vienen descritos en las pagínas \textsl{http://www.w3.org/TR/xmlschema11-2/} y \textsl{http://www.w3.org/TR/rdf-schema/}

	\lstinputlisting[language=owlms,literate=,linerange=Datatype:\ xsd:decimal-Datatype:\ rdfs:Literal]{t-box.ms} 
	

\section{Instancia de un título: Grado en Informática por la UPM}
A continuación vamos clasificar el título de grado en informática ofertado por la Universidad Politécnica de Madrid utilizando la ontología antes descrita, mediante la inclusión en la misma de los individuos que la componen. De este modo comprobaremos cómo la ontología es capaz de clasificar a todos los individuos que componen el título, además de comprobar la decidibilidad del conjunto. Este modelo del plan de estudios, nos permitirá más adelante (si está correctmente construido), la utilización de herramientas automáticas para profundizar en el análisis del plan de estudios que permitan encontrar inconsistencias en su diseño. 


\subsection{Individuos de la ontología}
	Los individuos representan objetos de la ontología en el dominio que estamos estudiando. Protegé no hace uso del Unique Name Assumption, es decir, para protegé dos individuos pueden referirse al mismo objeto del mundo real, salvo que se especifique lo contrario. Esta es una consecuencia de las ontologías OWL: todo lo que no sea dicho de forma explícita puede ser cierto. El hecho de que no especifiquemos si dos individuos son o no los mismos, significa que pueden o no serlo, para ese dominio. 
  
	En nuestra ontología todos los individuos que componen la ontología (11 \lstinline!Objetivo_General!, 12 \lstinline!Competencia_General!, 47 \lstinline!Competencia_Especifica!, 12 \lstinline!Materia!, 43 \lstinline!Asignaturas!, 66 \lstinline!Actividades_Formativas!\todo{arregla esto}, 6 \lstinline!Metodos_Docentes!, 4 \lstinline!Caracter!, 9 \lstinline!Metodo_Evaluacion!, 7 \lstinline!Ubicacion_Temporal!) son distintos. Dado el gran número de individuos que componen la ontología, se ha optado por no poner la declaración de todos y cada uno de ellos, si no tan sólo de algunos de ellos, a modo de ejemplo.
 
  %\lstinputlisting[language=owlms,literate=,linerange=DifferentIndividuals:-UT_7o_semestre]{a-box.ms}
  
   %\todo{decir que son distintos, y que no se pone para no extendernos se nombran algunos y punto. decir cuántos son}
  
  
	\begin{description}


		%%%%%%%%%%%%%%%%%%%%%%%%%%%%%%%%%%%%%%%%%%%%%%%%%%%%%%%%%%%%%%%%%
	   	%% Objetivos Generales
   		%%%%%%%%%%%%%%%%%%%%%%%%%%%%%%%%%%%%%%%%%%%%%%%%%%%%%%%%%%%%%%%%%    
		\item [Objetivos Generales]. Son individuos pertenecientes a la clase \lstinline!Objetivo_General!, que representan los objetivos generales del título, desde el número 1 al 11. A modo de ejemplo:
	 	\lstinputlisting[language=owlms,literate=,linerange=Individual:\ OBJ01-xsd:string]{a-box.ms}

		\lstinputlisting[language=owlms,literate=,linerange=Individual:\ OBJ02-xsd:string]{a-box.ms}
 		
	 
	 	%%%%%%%%%%%%%%%%%%%%%%%%%%%%%%%%%%%%%%%%%%%%%%%%%%%%%%%%%%%%%%%%%
   		%% Competencias Generales
	   	%%%%%%%%%%%%%%%%%%%%%%%%%%%%%%%%%%%%%%%%%%%%%%%%%%%%%%%%%%%%%%%%%    
		\item [Competencias Generales.] Estos son todos los individuos pertenecientes a la clase \lstinline!Competencia_General!, que representan las competencias generales que el alumno adquiere durante el itinerario formativo. Son las siguientes: \lstinline!CG-1-21!, \lstinline!CG-2-CE45!, \lstinline!CG-3-4!, \lstinline!CG-5!, \lstinline!CG-6!, \lstinline!CG-7-8-9-10-16-17!, \lstinline!CG-11-12-20!, \lstinline!CG-13-CE55!, \lstinline!CG-14-15-18-23!, \lstinline!CG-19!, \lstinline!CG-22! y \lstinline!CG-24-25-26-27!. A modo de ejemplo se incluyen:
		\lstinputlisting[language=owlms,literate=,linerange=Individual:\ CG\-1\-21-xsd:string]{a-box.ms}
 	 	\lstinputlisting[language=owlms,literate=,linerange=Individual:\ CG\-11\-12\-20-xsd:string]{a-box.ms}
 	 	\lstinputlisting[language=owlms,literate=,linerange=Individual:\ CG\-13\-CE55-xsd:string]{a-box.ms}
 	

		%%%%%%%%%%%%%%%%%%%%%%%%%%%%%%%%%%%%%%%%%%%%%%%%%%%%%%%%%%%%%%%%%
		%% Competencias Específicas
   		%%%%%%%%%%%%%%%%%%%%%%%%%%%%%%%%%%%%%%%%%%%%%%%%%%%%%%%%%%%%%%%%%    
		\item [Competencias Específicas]. Estos individuos pertentecientes a la clase \lstinline!Competencia_Especifica!, representan las diferentes competencias específicas que el alumno debe poseer al finalizar el ciclo formativo. En la titulación que estamos definiendo son: \lstinline!CE-1!, \lstinline!CE-2!, \lstinline!CE-3-4!, \lstinline!CE-5!, \lstinline!CE-6!, \lstinline!CE-7!, \lstinline!CE-8!, \lstinline!CE-9!, \lstinline!CE-10!, \lstinline!CE-12-16!, \lstinline!CE-13-18!, \lstinline!CE-14-15!, \lstinline!CE-17!, \lstinline!CE-19-20!, \lstinline!CE-21!, \lstinline!CE-22!, \lstinline!CE-23!, \lstinline!CE-24!, \lstinline!CE-25!, \lstinline!CE-26-27!, \lstinline!CE-28!, \lstinline!CE-29!, \lstinline!CE-30!, \lstinline!CE-31!, \lstinline!CE-32!, \lstinline!CE-33!, \lstinline!CE-34!, \lstinline!CE-35!, \lstinline!CE-36!, \lstinline!CE-37!, \lstinline!CE-38!, \lstinline!CE-39!, \lstinline!CE-40!, \lstinline!CE-41!, \lstinline!CE-42!, \lstinline!CE-43!, \lstinline!CE-44!, \lstinline!CE-46!, \lstinline!CE-47!, \lstinline!CE-48!, \lstinline!CE-49!, \lstinline!CE-50!, \lstinline!CE-51!, \lstinline!CE-52!, \lstinline!CE-53-54! y \lstinline!CE-56!.
 		\lstinputlisting[language=owlms,literate=,linerange=Individual:\ CE\-19\-20-xsd:string]{a-box.ms}
		\lstinputlisting[language=owlms,literate=,linerange=Individual:\ CE\-33-xsd:string]{a-box.ms}
 
 
		%%%%%%%%%%%%%%%%%%%%%%%%%%%%%%%%%%%%%%%%%%%%%%%%%%%%%%%%%%%%%%%%%
		%% Materias
		%%%%%%%%%%%%%%%%%%%%%%%%%%%%%%%%%%%%%%%%%%%%%%%%%%%%%%%%%%%%%%%%%    
		\item [Materias]. Estos individuos representan las diferentes materias de la titulación que estamos modelando Son: \lstinline!MA-Empresa!, \lstinline!MA-English_\text{for}_professional_and_academic_communication!, \lstinline!MA-Estadística!, \lstinline!MA-Física!, \lstinline!MA-Informática!, \lstinline!MA-Ingeniería_de_computadores!, \lstinline!MA-Ingeniería_del_software_sistemas_de_información_y_sistemas_inteligentes!, \lstinline!MA-Matemáticas!, \lstinline!MA-Optatividad!, \lstinline!MA-Programación!, \lstinline!MA-Sistemas_operativos,_sistemas_distribuidos_y_redes! y \lstinline!MA-Trabajo_fin_de_grado!. De igual modo, sólo se incluye una, a modo de ejemplo ilustrativo.
		\lstinputlisting[language=owlms,literate=,linerange=Individual:\ MA\-Ingeniería\_del\_software\_sistemas\_de\_información\_y\_sistemas\_inteligentes-xsd:decimal]{a-box.ms} 
	
	%\todo{dejar sólo una de ellas}
	
 		 

		%%%%%%%%%%%%%%%%%%%%%%%%%%%%%%%%%%%%%%%%%%%%%%%%%%%%%%%%%%%%%%%%%
	   	%% Asignaturas
	   	%%%%%%%%%%%%%%%%%%%%%%%%%%%%%%%%%%%%%%%%%%%%%%%%%%%%%%%%%%%%%%%%%     		 
		\item [Asignaturas]. Estos individuos (43 en total) son la representación de las diferentes asignaturas del plan de estudios. A modo de ejemplo, detallamos los siguientes.
		\lstinputlisting[language=owlms,literate=,linerange=Individual:\ AS\-Programación_II-xsd:string]{a-box.ms} 
		\lstinputlisting[language=owlms,literate=,linerange=Individual:\ AS\-Matemática_Discreta_II-xsd:string]{a-box.ms} 
		\lstinputlisting[language=owlms,literate=,linerange=Individual:\ AS\-Proyecto_de_Software-xsd:decimal]{a-box.ms} 
  
  
		%%%%%%%%%%%%%%%%%%%%%%%%%%%%%%%%%%%%%%%%%%%%%%%%%%%%%%%%%%%%%%%%%
	   	%% Métodos Docentes
   		%%%%%%%%%%%%%%%%%%%%%%%%%%%%%%%%%%%%%%%%%%%%%%%%%%%%%%%%%%%%%%%%%    
		\item [Métodos Docentes]. Aquí se agrupan los diferentes métodos docentes utilizados para la enseñanza de las materias. Se han instanciado los seis métodos docentes utilizados en el plan de estudios, que son: 
  
		\begin{description}
			\item [Aprendizaje basado en problemas.] El punto de partida de este método docente es el diseño de un problema por parte del profesor que el estudiante ha de resolver, para desarrollar unas competencias previamente definidas. El aprendizaje resulta más estimulante ya que requiere un mayor esfuerzo intelectual, y le permite al alumno experimentar e indagar sobre situaciones concretas del mundo real.
			\item [Aprendizaje Cooperativo.] Se trata de hacer que los alumnos sean responsables de su aprendizaje y del de sus compañeros que mediante estrategias de cooperatividad, permitan alcanzar metas e incentivos grupales. Prioriza la cooperación frente a la competición, y permite la adquisción de competencias para la interacción entre iguales y la adquisición de actitudes y valores.
			\item [Aprendizaje orientado a proyectos.] En este método los estudiantes llevan a cabo la realización de un proyecto en un tiempo determinado con el fin de resolver un problema o abordar una tarea mediante el diseño y realización de una serie de actividades, aplicando para ello los conocimientos adquiridos y utilizando los recursos disponibles de una manera efectiva.
			\item [Estudio de Casos.] Se trata de analizar de manera intensivaun problema o suceso real con la finalidad de conocerlo y poder generar hipótesis, resolverlo, intrepertarlo...  y con ello adquirir entrenamiento en los procedimientos de tratamiento y resolución.
			\item [Lección magistral.] En la lección magistral el profesor expone verbalmente el contenido de la materia a sus alumnos.
			\item [Resolución de ejercicios y problemas.] La base de estos métodos consiste el desarrollo por parte de los alumnos de soluciones adecuadas o correctas mediante la ejecución de rutinas y la apliación de procedimientos y una posterior interpretación de resultados.
		\end{description}

		A modo de ejemplo, mostramos un par de ellos.
    
		\lstinputlisting[language=owlms,literate=,linerange=Individual:\ ME\-Lección_magistral-xsd:string]{a-box.ms} 
  		\lstinputlisting[language=owlms,literate=,linerange=Individual:\ ME\-Aprendizaje_orientado_a_proyectos-xsd:string]{a-box.ms} 
  
  
		%%%%%%%%%%%%%%%%%%%%%%%%%%%%%%%%%%%%%%%%%%%%%%%%%%%%%%%%%%%%%%%%%
		%% Actividades Formativas
	   	%%%%%%%%%%%%%%%%%%%%%%%%%%%%%%%%%%%%%%%%%%%%%%%%%%%%%%%%%%%%%%%%%    
		\item [Actividades Formativas]. Estas son las diferentes actividades formativas definidas en el plan de estudios. Como ya se dijo anteriormente, con el fin de poder almacenar la información relativa a los créditos otorgados por cada actividad formativa, se ha definido un individuo \lstinline$actividad \text{formativa}$ \todo{arregla lo que pone formativa}por cada actividad formativa utilizada en cada materia. Esto hace que el número de individuos sea bastante elevado ($\#Actividades\_Formativas=\#Actividades \text{ } \times \text{ } \#Materias$), incluso después de eliminar aquellas actividades formativas que no se emplean en la docencia de una asignatura.
  
		%\todo{Habla de las actividades formativas de esta titulacion}
  		En este plan de estudios se utilizan nueve activiades formativas:
  
		\begin{description}
			\item [Clases Prácticas.] En las clases prácticas se desarrollan habilidades de aplicación de los conocimientos a situaciones concretas y adquisición de habilidades y procedimientos relacionados con la materia objeto de estudio. Se desarrolla en recintos en aulas y se organizan en grupos de tamaño medio o pequeño. Exigen la presencia de profesor y alumnos.
			\item [Clases Teóricas.] Básicamente son aquellas actividades que se centran en la exposición verbal de los contenidos por parte del profesor, de forma unidireccional, y sin que los alumnos intervengan en la selección de los contenidos.
			\item [Estudio y trabajo en grupo.] Es un enfoque cooperativo en el que cada alumno aprende de los demás, así como de su profesor y de su entorno. El éxito individual de cada alumno depende de que el conjunto de los compañeros alcancen las metas fijadas, y este éxito del grupo depende a su vez del desaroolo y despliegue de competencias sociales, claves en el desempeño profesional.
			\item [Estudio y trabajo autónomo individual.] Se trata de una actividad formativa en la que el estudiante se responsabiliza de la organización del trabajo y de la adquisición de las diferetnes competencias según su propio ritmo. Implica que el alumno debe asumir la responsabilidad y el control sobre su propio proceso de aprendizaje, tomando las decisiones sobre planificación, ejecución y evaluación del aprendizaje.
			\item [Clase de laboratorio.] Muy similar a las clases prácticas, con las misma metas y desarrollos, pero en lugar de desarrollarse en un aula común, se suceden en aulas de laboratorio, especialmente acondicionadas para la clase práctica. Al igual que el resto de clases prácticas, exigen la presencia de alumno y profesor.
			\item [Prácticas individuales o en grupo.] Son trabajos llevados a cabo de manera individual o en grupo, en los que el alumno debe resolver un problema planteado por el profesor, o bien diseñar y realizar una serie de actividades con el fin de poder realizar una tarea, analizando los resultados obtenidos.
			\item [Proyectos.] La realización y presentación de proyectos software o de instalaciones implica la realización de cálculos, recogida de datos, realización de pruebas, ensayos, demostraciones... que concluye en una demostración de las competencias adquiridas mediante la ejecución y demostración de tareas reales o simuladas.
			\item [Seminarios y Talleres.] Se llama así al espacio donde se construye con profundidad una parte del conocimiento a través de intercambios personales entre sus asistentes, cuyo número deberá ser reducido. Esta participación activa de todos los integrantes supone la elaboración previa de materiales y el acuerdo de unas normas para el desarrollo de la actividad. 
			\item [Tutorías.] En las tutorías se establece una relación personalizada de ayuda entre el profesor o tutor y el alumno. El tutor facilita y orienta al alumno en su aprendizaje. En una tutoría cabe tratar cualquier faceta del proceso de aprendizaje, como aspectos académicos, actitudinales, u otros.
		\end{description}
		\lstinputlisting[language=owlms,literate=,linerange=Individual:\ AF\-Proyecto\-PO-xsd:decimal]{a-box.ms} 
		\lstinputlisting[language=owlms,literate=,linerange=Individual:\ AF\-Fisica\-L-xsd:decimal]{a-box.ms} 
  
		%%%%%%%%%%%%%%%%%%%%%%%%%%%%%%%%%%%%%%%%%%%%%%%%%%%%%%%%%%%%%%%%%
	   	%% Métodos de Evaluación
	   	%%%%%%%%%%%%%%%%%%%%%%%%%%%%%%%%%%%%%%%%%%%%%%%%%%%%%%%%%%%%%%%%%    
		\item [Métodos de Evaluacion]. Estos son algunos de los métodos de evaluación utilizados a la hora de comprobar la asquisición de competencias por parte de los alumnos al finalizar el curso de las asignaturas. En el plan de estudios que nos ocupa, se han definido los siguientes:
  
		\begin{enumerate}
			\item Escalas de actitudes. Útiles para recoger opiniones, valores, habilidades sociales y directivas y conductas de interacción.
			\item Presentación de informes.
			\item Presentación de trabajos. 
			\item Pruebas de ejecución. Indisitantemente se pueden usar tareas reales o simuladas.
			\item Pruebas de respuesta corta. 
			\item Pruebas de respuesta larga, en las que el alumno debera argumentar y desarrollar una respuesta a la pregunta planteada por el profesor.
			\item Pruebas orales, bien en grupo, bien de forma individual, y donde se incluye la presentación de temas-trabajos.
			\item Sistemas de autoevaluación (oral, escrita o en grupo).
			\item Evaluación para el Prácticum. Recogida de informes/encuestas y defensa oral del trabajo realizado.
			\item Evaluación para los programas de movilidad. Recogida de informes, justificantes, calificaciones, realización de memoria y defensa oral.
		\end{enumerate}
		\lstinputlisting[language=owlms,literate=,linerange=Individual:\ MV\-Pruebas_de_ejecucion-xsd:string]{a-box.ms} 
		\lstinputlisting[language=owlms,literate=,linerange=Individual:\ MV\-Sistemas_de_Autoevaluacion-xsd:string]{a-box.ms} 
  
  
		%%%%%%%%%%%%%%%%%%%%%%%%%%%%%%%%%%%%%%%%%%%%%%%%%%%%%%%%%%%%%%%%%
	   	%% Carácter
   		%%%%%%%%%%%%%%%%%%%%%%%%%%%%%%%%%%%%%%%%%%%%%%%%%%%%%%%%%%%%%%%%%    
		\item [Carácter]. Aquí quedan recogidos los diferentes carácteres que asignaturas y materias pueden tener. Una asignatura puede ser básica, obligatoria u optativa, mientras que una materia puede ser, además, de carácter mixto.
		\lstinputlisting[language=owlms,literate=,linerange=Individual:\ CA\-Básica-TBox:Caracter]{a-box.ms} 
		\lstinputlisting[language=owlms,literate=,linerange=Individual:\ CA\-Mixta-TBox:Caracter]{a-box.ms} 
		\lstinputlisting[language=owlms,literate=,linerange=Individual:\ CA\-Obligatoria-TBox:Caracter]{a-box.ms} 
		\lstinputlisting[language=owlms,literate=,linerange=Individual:\ CA\-Optativa-TBox:Caracter]{a-box.ms} 
  
		%%%%%%%%%%%%%%%%%%%%%%%%%%%%%%%%%%%%%%%%%%%%%%%%%%%%%%%%%%%%%%%%%
		%% Ubicaciones Temporales
		%%%%%%%%%%%%%%%%%%%%%%%%%%%%%%%%%%%%%%%%%%%%%%%%%%%%%%%%%%%%%%%%%  
		\item [Ubicaciones temporales]. Aqui se agrupan, con el fin de poder lograr un correcto reparto de las asiganturas entre los trimestres que componen la titulación, los trimestres en los que se divide el título. A modo enunciativo:
		\lstinputlisting[language=owlms,literate=,linerange=Individual:\ UT\-3er_semestre-TBox:Ubicacion_temporal]{a-box.ms} 
		\lstinputlisting[language=owlms,literate=,linerange=Individual:\ UT\-5o_semestre-TBox:Ubicacion_temporal]{a-box.ms} 

	\end{description}

%%% Local Variables: 
%%% mode: latex
%%% TeX-master: "tfc-ontologia-grado"
%%% TeX-PDF-mode: t
%%% ispell-local-dictionary: "castellano"
%%% End: 
